
Uit onze resultaten kan geconcludeerd worden dat de vertraging van een klokbuffer circuit met enkele inverter groter is dan bij een klokbuffer circuit met meerdere inverters in cascade. Dit is belangrijk om te weten bij het kiezen een klokbuffer bij een bepaalde belastingscapaciteit. 
Voordelen cascade inverters tegenover enkele inverter.
\begin{enumerate}
\item De vertragingstijd is kleiner dus zal er eerder een 1 of een 0 op de uitgang van de buffer staan, zo kan de rest van het circuit daar weer sneller op reageren, dus de totale vertragingstijd van een bepaald circuit daalt.
\item Inverters in cascade hebben een herstellend effect waardoor ook een zwakkere 1 of een zwakkere 0 een sterke 1 of sterke 0 als uitgang hebben. 
\end {enumerate}
Nadelen cascade inverters tegenover enkele inverter.
\begin{enumerate}
\item Het is natuurlijk duurder, omdat je meerdere inverters nodig hebt die dan gafabriceerd moeten worden.
\item De stroom die nodig is, stijgt door de interne weerstand van de inverters.
\end {enumerate}

Dus bij het kiezen van een klokbuffer moet gekeken worden naar de grote van de belastingscapaciteit. Als het een kleine capaciteit is dan is een klokbuffer met enkele inverter voldoende, want de verminderde vertraging is minimaal dus is goedkoper een betere keuze.
Als de capaciteit groot is en de vertraging een beperkende factor is dan is het beter om te kiezen voor de inverters in cascade, omdat de vertraging bij een grotere capaciteit ook steeds groter wordt, dus dan zal de vertragings vermindering van de inverters in cascade een grote invloed hebben en zal het de kosten van de extra inverters zeker waard zijn.   