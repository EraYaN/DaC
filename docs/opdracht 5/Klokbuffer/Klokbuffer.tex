\documentclass{report}
% Include all project wide packages here.
\usepackage{fullpage}
\usepackage{polyglossia}
\setmainlanguage{dutch}
\usepackage{csquotes}
\usepackage{graphicx}
\usepackage{epstopdf}
\usepackage{pdfpages}
\usepackage{caption}
\usepackage[list=true]{subcaption}
\usepackage{float}
\usepackage{standalone}
\usepackage{import}
\usepackage{tocloft}
\usepackage{wrapfig}
\usepackage{authblk}
\usepackage{array}
\usepackage{booktabs}
\usepackage[toc,page,title,titletoc]{appendix}
\usepackage{xunicode}
\usepackage{fontspec}
\usepackage{pgfplots}
\usepackage{SIunits}
\pgfplotsset{compat=newest}
\pgfplotsset{plot coordinates/math parser=false}
\newlength\figureheight 
\newlength\figurewidth
\usepackage{unicode-math}
\usepackage[
    backend=bibtexu,
	texencoding=utf8,
bibencoding=utf8,
    style=ieee,
    sortlocale=nl_NL,
    language=auto
]{biblatex}
\usepackage{listings}
\newcommand{\includecode}[3][c]{\lstinputlisting[caption=#2, escapechar=, style=#1]{#3}}
\newcommand{\superscript}[1]{\ensuremath{^{\textrm{#1}}}}
\newcommand{\subscript}[1]{\ensuremath{_{\textrm{#1}}}}


\newcommand{\chapternumber}{\thechapter}
\renewcommand{\appendixname}{Bijlage}
\renewcommand{\appendixtocname}{Bijlagen}
\renewcommand{\appendixpagename}{Bijlagen}

\usepackage[hidelinks]{hyperref} %<--------ALTIJD ALS LAATSTE

%\renewcommand{\familydefault}{\sfdefault}

\setmainfont[Ligatures=TeX]{Myriad Pro}
\setmathfont{Asana Math}
\setmonofont{Lucida Console}

\usepackage{titlesec, blindtext, color}
\definecolor{gray75}{gray}{0.75}
\newcommand{\hsp}{\hspace{20pt}}
\titleformat{\chapter}[hang]{\Huge\bfseries}{\chapternumber\hsp\textcolor{gray75}{|}\hsp}{0pt}{\Huge\bfseries}
\renewcommand{\familydefault}{\sfdefault}
\renewcommand{\arraystretch}{1.2}
\setlength\parindent{0pt}

%For code listings
\definecolor{black}{rgb}{0,0,0}
\definecolor{browntags}{rgb}{0.65,0.1,0.1}
\definecolor{bluestrings}{rgb}{0,0,1}
\definecolor{graycomments}{rgb}{0.4,0.4,0.4}
\definecolor{redkeywords}{rgb}{1,0,0}
\definecolor{bluekeywords}{rgb}{0.13,0.13,0.8}
\definecolor{greencomments}{rgb}{0,0.5,0}
\definecolor{redstrings}{rgb}{0.9,0,0}
\definecolor{purpleidentifiers}{rgb}{0.01,0,0.01}


\lstdefinestyle{csharp}{
language=[Sharp]C,
showspaces=false,
showtabs=false,
breaklines=true,
showstringspaces=false,
breakatwhitespace=true,
escapeinside={(*@}{@*)},
columns=fullflexible,
commentstyle=\color{greencomments},
keywordstyle=\color{bluekeywords}\bfseries,
stringstyle=\color{redstrings},
identifierstyle=\color{purpleidentifiers},
basicstyle=\ttfamily\small}

\lstdefinestyle{c}{
language=C,
showspaces=false,
showtabs=false,
breaklines=true,
showstringspaces=false,
breakatwhitespace=true,
escapeinside={(*@}{@*)},
columns=fullflexible,
commentstyle=\color{greencomments},
keywordstyle=\color{bluekeywords}\bfseries,
stringstyle=\color{redstrings},
identifierstyle=\color{purpleidentifiers},
}

\lstdefinestyle{matlab}{
language=Matlab,
showspaces=false,
showtabs=false,
breaklines=true,
showstringspaces=false,
breakatwhitespace=true,
escapeinside={(*@}{@*)},
columns=fullflexible,
commentstyle=\color{greencomments},
keywordstyle=\color{bluekeywords}\bfseries,
stringstyle=\color{redstrings},
identifierstyle=\color{purpleidentifiers}
}

\lstdefinestyle{vhdl}{
language=VHDL,
showspaces=false,
showtabs=false,
breaklines=true,
showstringspaces=false,
breakatwhitespace=true,
escapeinside={(*@}{@*)},
columns=fullflexible,
commentstyle=\color{greencomments},
keywordstyle=\color{bluekeywords}\bfseries,
stringstyle=\color{redstrings},
identifierstyle=\color{purpleidentifiers}
}

\lstdefinestyle{xaml}{
language=XML,
showspaces=false,
showtabs=false,
breaklines=true,
showstringspaces=false,
breakatwhitespace=true,
escapeinside={(*@}{@*)},
columns=fullflexible,
commentstyle=\color{greencomments},
keywordstyle=\color{redkeywords},
stringstyle=\color{bluestrings},
tagstyle=\color{browntags},
morestring=[b]",
  morecomment=[s]{<?}{?>},
  morekeywords={xmlns,version,typex:AsyncRecords,x:Arguments,x:Boolean,x:Byte,x:Char,x:Class,x:ClassAttributes,x:ClassModifier,x:Code,x:ConnectionId,x:Decimal,x:Double,x:FactoryMethod,x:FieldModifier,x:Int16,x:Int32,x:Int64,x:Key,x:Members,x:Name,x:Object,x:Property,x:Shared,x:Single,x:String,x:Subclass,x:SynchronousMode,x:TimeSpan,x:TypeArguments,x:Uid,x:Uri,x:XData,Grid.Column,Grid.ColumnSpan,Click,ClipToBounds,Content,DropDownOpened,FontSize,Foreground,Header,Height,HorizontalAlignment,HorizontalContentAlignment,IsCancel,IsDefault,IsEnabled,IsSelected,Margin,MinHeight,MinWidth,Padding,SnapsToDevicePixels,Target,TextWrapping,Title,VerticalAlignment,VerticalContentAlignment,Width,WindowStartupLocation,Binding,Mode,OneWay,xmlns:x}
}

%defaults
\lstset{
basicstyle=\ttfamily\small,
extendedchars=false,
numbers=left,
numberstyle=\ttfamily\tiny,
stepnumber=1,
tabsize=4,
numbersep=5pt
}
\usepackage{fullpage}

\begin{document}
\newcommand{\rp}{$\rightarrow$}
\newcommand{\Ohm}{$\Omega$}
\newcommand{\ohm}{$\omega$}
\newcommand{\gmu}{$\mu$}
\newcommand{\tss}{\textsubscript}
\newcommand{\ci}{\lstinputlisting} % code input = ci

%%%%%%%%%%%%%%%%%%%%%%%%%%%%%%%%%%%%%%%%%
% University Assignment Title Page 
% LaTeX Template
% Version 1.0 (27/12/12)
%
% This template has been downloaded from:
% http://www.LaTeXTemplates.com
%
% Original author:
% WikiBooks (http://en.wikibooks.org/wiki/LaTeX/Title_Creation)
%
% License:
% CC BY-NC-SA 3.0 (http://creativecommons.org/licenses/by-nc-sa/3.0/)
% 
% Instructions for using this template:
% This title page is capable of being compiled as is. This is not useful for 
% including it in another document. To do this, you have two options: 
%
% 1) Copy/paste everything between \begin{document} and \end{document} 
% starting at \begin{titlepage} and paste this into another LaTeX file where you 
% want your title page.
% OR
% 2) Remove everything outside the \begin{titlepage} and \end{titlepage} and 
% move this file to the same directory as the LaTeX file you wish to add it to. 
% Then add %%%%%%%%%%%%%%%%%%%%%%%%%%%%%%%%%%%%%%%%%
% University Assignment Title Page 
% LaTeX Template
% Version 1.0 (27/12/12)
%
% This template has been downloaded from:
% http://www.LaTeXTemplates.com
%
% Original author:
% WikiBooks (http://en.wikibooks.org/wiki/LaTeX/Title_Creation)
%
% License:
% CC BY-NC-SA 3.0 (http://creativecommons.org/licenses/by-nc-sa/3.0/)
% 
% Instructions for using this template:
% This title page is capable of being compiled as is. This is not useful for 
% including it in another document. To do this, you have two options: 
%
% 1) Copy/paste everything between \begin{document} and \end{document} 
% starting at \begin{titlepage} and paste this into another LaTeX file where you 
% want your title page.
% OR
% 2) Remove everything outside the \begin{titlepage} and \end{titlepage} and 
% move this file to the same directory as the LaTeX file you wish to add it to. 
% Then add %%%%%%%%%%%%%%%%%%%%%%%%%%%%%%%%%%%%%%%%%
% University Assignment Title Page 
% LaTeX Template
% Version 1.0 (27/12/12)
%
% This template has been downloaded from:
% http://www.LaTeXTemplates.com
%
% Original author:
% WikiBooks (http://en.wikibooks.org/wiki/LaTeX/Title_Creation)
%
% License:
% CC BY-NC-SA 3.0 (http://creativecommons.org/licenses/by-nc-sa/3.0/)
% 
% Instructions for using this template:
% This title page is capable of being compiled as is. This is not useful for 
% including it in another document. To do this, you have two options: 
%
% 1) Copy/paste everything between \begin{document} and \end{document} 
% starting at \begin{titlepage} and paste this into another LaTeX file where you 
% want your title page.
% OR
% 2) Remove everything outside the \begin{titlepage} and \end{titlepage} and 
% move this file to the same directory as the LaTeX file you wish to add it to. 
% Then add \input{./title_page_1.tex} to your LaTeX file where you want your
% title page.
%
%%%%%%%%%%%%%%%%%%%%%%%%%%%%%%%%%%%%%%%%%

%----------------------------------------------------------------------------------------
%	PACKAGES AND OTHER DOCUMENT CONFIGURATIONS
%----------------------------------------------------------------------------------------


\documentclass[12pt]{article}
\newcommand{\tss}{\textsubscript}
\input{../../../library/preamble}

\begin{document}


\begin{titlepage}

\newcommand{\HRule}{\rule{\linewidth}{0.5mm}} % Defines a new command for the horizontal lines, change thickness here

\center % Center everything on the page
 
%----------------------------------------------------------------------------------------
%	HEADING SECTIONS
%----------------------------------------------------------------------------------------

\textsc{\LARGE TU Delft}\\[1.5cm] % Name of your university/college
\textsc{\Large EE28 EPO-3 Moduleopdracht 5}\\[0.5cm] % Major heading such as course name
\textsc{\large Klokbuffer}\\[0.5cm] % Minor heading such as course title

%----------------------------------------------------------------------------------------
%	TITLE SECTION
%----------------------------------------------------------------------------------------

\HRule \\[0.4cm]
{ \huge \bfseries Klokbuffer  noodzakelijk voor kloknetwerk?}\\[0.4cm] % Title of your document
\HRule \\[1.5cm]
 
%----------------------------------------------------------------------------------------
%	AUTHOR SECTION
%----------------------------------------------------------------------------------------

\begin{minipage}{0.4\textwidth}
\begin{flushleft} \large
\emph{Author:}\\
Kees \textsc{Hogenhout} 4220927\\ % Your name 
Tu \textsc{Hoang} 4203496 %other name
\end{flushleft}
\end{minipage}
~
\begin{minipage}{0.4\textwidth}
\begin{flushright} \large
\emph{Supervisor:} \\
Dr. Marjam \textsc{Bartek} % Supervisor's Name
\end{flushright}
\end{minipage}\\[4cm]

% If you don't want a supervisor, uncomment the two lines below and remove the section above
%\Large \emph{Author:}\\
%John \textsc{Smith}\\[3cm] % Your name

%----------------------------------------------------------------------------------------
%	DATE SECTION
%----------------------------------------------------------------------------------------

{\large \today}\\[3cm] % Date, change the \today to a set date if you want to be precise

%----------------------------------------------------------------------------------------
%	LOGO SECTION
%----------------------------------------------------------------------------------------

%\includegraphics{Logo}\\[1cm] % Include a department/university logo - this will require the graphicx package
 
%----------------------------------------------------------------------------------------
% 

\textbf{Abstract} \\
In dit onderzoekje wordt uitgezocht in welke mate een klokbuffer nodig is voor een netwerk van 125 D-flipflops. Eerst wordt de t-90\% van een inverter bepaald met een C\tss{load} van 25pF (belastingscapaciteit van die 125 DFF's). Vervolgens worden er drie inverters (die groter zijn)  in cascade geplaatst met dezelfde belastingscapaciteit en wordt weer de t-90\% bepaald. Te zien is aan de simulatieresultaten dat de drie inverters als buffer werken. De afvlakking bij een 0-1 overgang van de uitgang is namelijk heel klein, terwijl die afvlakking bij de meting met een enkele transistor veel groter is. 

\vfill % Fill the rest of the page with whitespace

\end{titlepage}
\end{document} to your LaTeX file where you want your
% title page.
%
%%%%%%%%%%%%%%%%%%%%%%%%%%%%%%%%%%%%%%%%%

%----------------------------------------------------------------------------------------
%	PACKAGES AND OTHER DOCUMENT CONFIGURATIONS
%----------------------------------------------------------------------------------------


\documentclass[12pt]{article}
\newcommand{\tss}{\textsubscript}
% Include all project wide packages here.
\usepackage{fullpage}
\usepackage{polyglossia}
\setmainlanguage{dutch}
\usepackage{csquotes}
\usepackage{graphicx}
\usepackage{epstopdf}
\usepackage{pdfpages}
\usepackage{caption}
\usepackage[list=true]{subcaption}
\usepackage{float}
\usepackage{standalone}
\usepackage{import}
\usepackage{tocloft}
\usepackage{wrapfig}
\usepackage{authblk}
\usepackage{array}
\usepackage{booktabs}
\usepackage[toc,page,title,titletoc]{appendix}
\usepackage{xunicode}
\usepackage{fontspec}
\usepackage{pgfplots}
\usepackage{SIunits}
\pgfplotsset{compat=newest}
\pgfplotsset{plot coordinates/math parser=false}
\newlength\figureheight 
\newlength\figurewidth
\usepackage{unicode-math}
\usepackage[
    backend=bibtexu,
	texencoding=utf8,
bibencoding=utf8,
    style=ieee,
    sortlocale=nl_NL,
    language=auto
]{biblatex}
\usepackage{listings}
\newcommand{\includecode}[3][c]{\lstinputlisting[caption=#2, escapechar=, style=#1]{#3}}
\newcommand{\superscript}[1]{\ensuremath{^{\textrm{#1}}}}
\newcommand{\subscript}[1]{\ensuremath{_{\textrm{#1}}}}


\newcommand{\chapternumber}{\thechapter}
\renewcommand{\appendixname}{Bijlage}
\renewcommand{\appendixtocname}{Bijlagen}
\renewcommand{\appendixpagename}{Bijlagen}

\usepackage[hidelinks]{hyperref} %<--------ALTIJD ALS LAATSTE


\begin{document}


\begin{titlepage}

\newcommand{\HRule}{\rule{\linewidth}{0.5mm}} % Defines a new command for the horizontal lines, change thickness here

\center % Center everything on the page
 
%----------------------------------------------------------------------------------------
%	HEADING SECTIONS
%----------------------------------------------------------------------------------------

\textsc{\LARGE TU Delft}\\[1.5cm] % Name of your university/college
\textsc{\Large EE28 EPO-3 Moduleopdracht 5}\\[0.5cm] % Major heading such as course name
\textsc{\large Klokbuffer}\\[0.5cm] % Minor heading such as course title

%----------------------------------------------------------------------------------------
%	TITLE SECTION
%----------------------------------------------------------------------------------------

\HRule \\[0.4cm]
{ \huge \bfseries Klokbuffer  noodzakelijk voor kloknetwerk?}\\[0.4cm] % Title of your document
\HRule \\[1.5cm]
 
%----------------------------------------------------------------------------------------
%	AUTHOR SECTION
%----------------------------------------------------------------------------------------

\begin{minipage}{0.4\textwidth}
\begin{flushleft} \large
\emph{Author:}\\
Kees \textsc{Hogenhout} 4220927\\ % Your name 
Tu \textsc{Hoang} 4203496 %other name
\end{flushleft}
\end{minipage}
~
\begin{minipage}{0.4\textwidth}
\begin{flushright} \large
\emph{Supervisor:} \\
Dr. Marjam \textsc{Bartek} % Supervisor's Name
\end{flushright}
\end{minipage}\\[4cm]

% If you don't want a supervisor, uncomment the two lines below and remove the section above
%\Large \emph{Author:}\\
%John \textsc{Smith}\\[3cm] % Your name

%----------------------------------------------------------------------------------------
%	DATE SECTION
%----------------------------------------------------------------------------------------

{\large \today}\\[3cm] % Date, change the \today to a set date if you want to be precise

%----------------------------------------------------------------------------------------
%	LOGO SECTION
%----------------------------------------------------------------------------------------

%\includegraphics{Logo}\\[1cm] % Include a department/university logo - this will require the graphicx package
 
%----------------------------------------------------------------------------------------
% 

\textbf{Abstract} \\
In dit onderzoekje wordt uitgezocht in welke mate een klokbuffer nodig is voor een netwerk van 125 D-flipflops. Eerst wordt de t-90\% van een inverter bepaald met een C\tss{load} van 25pF (belastingscapaciteit van die 125 DFF's). Vervolgens worden er drie inverters (die groter zijn)  in cascade geplaatst met dezelfde belastingscapaciteit en wordt weer de t-90\% bepaald. Te zien is aan de simulatieresultaten dat de drie inverters als buffer werken. De afvlakking bij een 0-1 overgang van de uitgang is namelijk heel klein, terwijl die afvlakking bij de meting met een enkele transistor veel groter is. 

\vfill % Fill the rest of the page with whitespace

\end{titlepage}
\end{document} to your LaTeX file where you want your
% title page.
%
%%%%%%%%%%%%%%%%%%%%%%%%%%%%%%%%%%%%%%%%%

%----------------------------------------------------------------------------------------
%	PACKAGES AND OTHER DOCUMENT CONFIGURATIONS
%----------------------------------------------------------------------------------------


\documentclass[12pt]{article}
\newcommand{\tss}{\textsubscript}
% Include all project wide packages here.
\usepackage{fullpage}
\usepackage{polyglossia}
\setmainlanguage{dutch}
\usepackage{csquotes}
\usepackage{graphicx}
\usepackage{epstopdf}
\usepackage{pdfpages}
\usepackage{caption}
\usepackage[list=true]{subcaption}
\usepackage{float}
\usepackage{standalone}
\usepackage{import}
\usepackage{tocloft}
\usepackage{wrapfig}
\usepackage{authblk}
\usepackage{array}
\usepackage{booktabs}
\usepackage[toc,page,title,titletoc]{appendix}
\usepackage{xunicode}
\usepackage{fontspec}
\usepackage{pgfplots}
\usepackage{SIunits}
\pgfplotsset{compat=newest}
\pgfplotsset{plot coordinates/math parser=false}
\newlength\figureheight 
\newlength\figurewidth
\usepackage{unicode-math}
\usepackage[
    backend=bibtexu,
	texencoding=utf8,
bibencoding=utf8,
    style=ieee,
    sortlocale=nl_NL,
    language=auto
]{biblatex}
\usepackage{listings}
\newcommand{\includecode}[3][c]{\lstinputlisting[caption=#2, escapechar=, style=#1]{#3}}
\newcommand{\superscript}[1]{\ensuremath{^{\textrm{#1}}}}
\newcommand{\subscript}[1]{\ensuremath{_{\textrm{#1}}}}


\newcommand{\chapternumber}{\thechapter}
\renewcommand{\appendixname}{Bijlage}
\renewcommand{\appendixtocname}{Bijlagen}
\renewcommand{\appendixpagename}{Bijlagen}

\usepackage[hidelinks]{hyperref} %<--------ALTIJD ALS LAATSTE


\begin{document}


\begin{titlepage}

\newcommand{\HRule}{\rule{\linewidth}{0.5mm}} % Defines a new command for the horizontal lines, change thickness here

\center % Center everything on the page
 
%----------------------------------------------------------------------------------------
%	HEADING SECTIONS
%----------------------------------------------------------------------------------------

\textsc{\LARGE TU Delft}\\[1.5cm] % Name of your university/college
\textsc{\Large EE28 EPO-3 Moduleopdracht 5}\\[0.5cm] % Major heading such as course name
\textsc{\large Klokbuffer}\\[0.5cm] % Minor heading such as course title

%----------------------------------------------------------------------------------------
%	TITLE SECTION
%----------------------------------------------------------------------------------------

\HRule \\[0.4cm]
{ \huge \bfseries Klokbuffer  noodzakelijk voor kloknetwerk?}\\[0.4cm] % Title of your document
\HRule \\[1.5cm]
 
%----------------------------------------------------------------------------------------
%	AUTHOR SECTION
%----------------------------------------------------------------------------------------

\begin{minipage}{0.4\textwidth}
\begin{flushleft} \large
\emph{Author:}\\
Kees \textsc{Hogenhout} 4220927\\ % Your name 
Tu \textsc{Hoang} 4203496 %other name
\end{flushleft}
\end{minipage}
~
\begin{minipage}{0.4\textwidth}
\begin{flushright} \large
\emph{Supervisor:} \\
Dr. Marjam \textsc{Bartek} % Supervisor's Name
\end{flushright}
\end{minipage}\\[4cm]

% If you don't want a supervisor, uncomment the two lines below and remove the section above
%\Large \emph{Author:}\\
%John \textsc{Smith}\\[3cm] % Your name

%----------------------------------------------------------------------------------------
%	DATE SECTION
%----------------------------------------------------------------------------------------

{\large \today}\\[3cm] % Date, change the \today to a set date if you want to be precise

%----------------------------------------------------------------------------------------
%	LOGO SECTION
%----------------------------------------------------------------------------------------

%\includegraphics{Logo}\\[1cm] % Include a department/university logo - this will require the graphicx package
 
%----------------------------------------------------------------------------------------
% 

\textbf{Abstract} \\
In dit onderzoekje wordt uitgezocht in welke mate een klokbuffer nodig is voor een netwerk van 125 D-flipflops. Eerst wordt de t-90\% van een inverter bepaald met een C\tss{load} van 25pF (belastingscapaciteit van die 125 DFF's). Vervolgens worden er drie inverters (die groter zijn)  in cascade geplaatst met dezelfde belastingscapaciteit en wordt weer de t-90\% bepaald. Te zien is aan de simulatieresultaten dat de drie inverters als buffer werken. De afvlakking bij een 0-1 overgang van de uitgang is namelijk heel klein, terwijl die afvlakking bij de meting met een enkele transistor veel groter is. 

\vfill % Fill the rest of the page with whitespace

\end{titlepage}
\end{document}
\tableofcontents
\chapter{Inleiding}
Bij het vak geintegreerde schakelingen werd behandeld dat de klok bij elke tik in die mate capaciteiten oplaad, dat dat de helft van de vermogensconsumptie van de chip vormt. Daar gaat het in dit rapport niet direct over, maar wel dat er tussen de klok en het circuit (C\tss{load}) een capaciteit zit, die opgeladen wordt door de klok. Die kan zorgen voor meer afvlakking van de voltage transfer characteristic. (VTC) In dit onderzoek wordt erop ingegaan of het plaatsen van een buffer tussen de klok en het netwerk (met C\tss{load} = 25pF) zorgt voor een kleinere t-90\%. \newline \\
De probleemstelling is net genoemd. Vervolgens is er theorie nodig om de begrippen, zoals t-90\% en afvlakking te begrijpen. Dit zal kort zijn, omdat het onderzoek klein is. Verder zal de methode beschreven worden, die daarna ook gevolgd is. Daarna worden de resultaten gepresenteerd en bediscussieerd. Als laatste wordt een conclusie gegeven. 

\chapter{Theorie}
In figuur \ref{Th1} staat de VTC van de inverter. Wanneer de ingang van 0-1 gaat, gaat de uitgang van 1-0. De t\tss{r} is de tijd waarin wij geïnteresseerd zijn. Daarin komen een t-10\% en een t-90\% voor. Dat zijn de punten waar de VTC een buigpunt heeft binnen t\tss{r}. 
\begin{figure} [h!]
\includegraphics [scale= 0.3] {inputfiles/vtc}
\caption{Voltage Transfer characteristic, bron: slides geïntegreerde schakelingen}
\label{Th1}
\end{figure}
De afvlakking wordt veroorzaakt doordat de capaciteit opgeladen moet worden. De verwachting is dat de afvlakking kleiner wordt door de buffer, omdat de bredere transistors meer stroom kunnen leveren, die de C\tss{load} sneller oplaadt. 



\chapter{Eisen}



De simulatieopdracht maakt gebruik van de transistorparameters die horen bij het gebruikte EPO-3 Sea-of-Gates image. Voor deze transistoren gelden de volgende randvoorwaarden:
\begin{itemize}
\item{Nominale \emph{V\textsubscript{DD}}:		5V}
\item{Minimale kanaallengte:		1.6$\mu$m}
\item{Maximale kanaallengte:		4.6$\mu$m}
\item{Minimale kanaalbreedte:		2.4$\mu$m}
\item{Maximale kanaalbreedte:		10000$\mu$m}
\end{itemize}
De PSPICE transistorparameters zijn volgens een zogenaamd \emph{level 2 model}. Zie paragraaf 3.3.4 van het Rabaey [1] boek.






\chapter{Methode}


Om de dikte van het gate-oxide te bepalen is een simulatie in PSPICE niet voldoende. Het bepalen van de dikte vereist daarom een combinatie van simulaties en berekeningen die als volgt worden beschreven:
\begin{enumerate}
\item{\textbf{Totale gate capaciteit \emph{C\tss{G}} bepalen d.m.v. PSPICE simulatie}}
\\
Als eerst moet de totale gate capaciteit \emph{C\tss{G}} worden bepaald. Om deze \emph{C\tss{G}} te bepalen moeten we eerst de stroom \emph{I} bepalen door het oxide en de spanning \emph{V\tss{GS}} over het oxide bepalen. De formule die de relatie beschrijft tussen de stroom, spanning en de capaciteit \emph{C\tss{G}} is:
\begin{equation}
I = C_g(V\tss{GS}) \frac{dV\tss{GS}}{dt}
\end{equation}
Als we deze formule omschrijven naar
 \begin{equation}
 C_g(V\tss{GS})= \frac{I}{ \frac{dV\tss{GS}}{dt}}
 \end{equation}
kunnen we de \emph{C\tss{G}} bepalen. Deze simulatie dient een aantal keren te worden herhaald waarbij de \emph{W = 3.0$\mu$m} en de \emph{L} varieert tussen \emph{2.2$\mu$m} en \emph{4.8$\mu$m}. Dit geldt zowel voor de NMOS schakeling als voor de PMOS schakeling. Uit deze simulaties ontstaan grafieken die de \emph{C\tss{G}} bepalen voor verschillende waarden van de gate-source spanning V\tss{GS}. We zijn geïnteresseerd in de waarde van \emph{C\tss{G}} op het meest vlakke gedeelte van de grafiek. Deze waardes worden verder gebruikt bij stap 2.
\\
\\In figuur \ref{res:PMOS_schakeling} is de opstelling van de PMOS simulatie te zien en in figuur \ref{res:NMOS_schakeling} is de opstelling van NMOS schakeling te zien. In beide opstellingen is de weerstand \emph{R3 = 5k\textohm} en de stroom van de stroombron \emph{I2 = 1mA}. Verder zijn er ook twee probes aangesloten om de stroom en de spanning te meten. De simulatietijd is 100ps.

 \begin{figure} [h!]
 \includegraphics [scale = 0.7] {figures/PMOS}
 \caption{Opstelling voor bepalen van de \emph{C\tss{G}} van de PMOS transistor}
 \label{res:PMOS_schakeling}
 \end{figure}

 \begin{figure} [h!]
 \includegraphics [scale = 0.7] {figures/NMOS}
 \caption{Opstelling voor bepalen van de \emph{C\tss{G}} van de NMOS transistor}
 \label{res:NMOS_schakeling}
 \end{figure}


\item{\textbf{Overlap capaciteit per lengte-eenheid \emph{C\tss{o}} bepalen m.b.v. MATLAB}}
\\
Als tweede stap gaan we de \emph{C\tss{o}} bepalen met behulp van de volgende formule:
\begin{equation}
C\tss{G} = W \ast C\tss{o} + L \ast W \ast C\tss{GC}
\end{equation}
Met behulp van MATLAB kunnen we de gevonden waarden van \emph{C\tss{G}} uit stap 1 met bijbehorende kanaallengtes \emph{L} invullen en zo een lineaire grafiek plotten. Als we deze grafiek doortrekken tot het punt waarbij \emph{L = 0}, kunnen we op de y-as de waarde uitlezen die gelijk is aan \emph{W \ast C\tss{o}}. Omdat \emph{W} vast staat kunnen we vervolgens \emph{C\tss{o}} bepalen.

\item{\textbf{Gate-channel capaciteit per oppervlakte-eenheid \emph{C\tss{GC}} berekenen}}
\\
Nadat de \emph{C\tss{G}} en de \emph{C\tss{o}} zijn bepaald kunnen we de \emph{C\tss{GC}} bepalen. Als we de formule uit stap 2 omschrijven krijgen we:
\begin{equation}
C\tss{GC} = \frac{C\tss{G} - C\tss{o} \ast W}{L \ast W}
\end{equation}
waaruit we \emph{C\tss{GC}} kunnen bepalen.


\item{\textbf{Dikte \emph{t\tss{ox}} van het gate oxide bepalen}}
\\
Als laatste stap kunnen we de dikte van het gate oxide bepalen aan de hand van de volgende formule:
\begin{equation}
t\tss{ox} = \frac{\epsilon\tss{ox}}{C\tss{GC}}
\end{equation}
Hierbij is \emph{$\epsilon$\tss{ox} = 35 fF/m}.



\end{enumerate}


\chapter{Resultaten}
\documentclass{scrartcl} % scrartcl of scrreprt
% Include all project wide packages here.
\usepackage{fullpage}
\usepackage{polyglossia}
\setmainlanguage{dutch}
\usepackage{csquotes}
\usepackage{graphicx}
\usepackage{epstopdf}
\usepackage{pdfpages}
\usepackage{caption}
\usepackage[list=true]{subcaption}
\usepackage{float}
\usepackage{standalone}
\usepackage{import}
\usepackage{tocloft}
\usepackage{wrapfig}
\usepackage{authblk}
\usepackage{array}
\usepackage{booktabs}
\usepackage[toc,page,title,titletoc]{appendix}
\usepackage{xunicode}
\usepackage{fontspec}
\usepackage{pgfplots}
\usepackage{SIunits}
\pgfplotsset{compat=newest}
\pgfplotsset{plot coordinates/math parser=false}
\newlength\figureheight 
\newlength\figurewidth
\usepackage{unicode-math}
\usepackage[
    backend=bibtexu,
	texencoding=utf8,
bibencoding=utf8,
    style=ieee,
    sortlocale=nl_NL,
    language=auto
]{biblatex}
\usepackage{listings}
\newcommand{\includecode}[3][c]{\lstinputlisting[caption=#2, escapechar=, style=#1]{#3}}
\newcommand{\superscript}[1]{\ensuremath{^{\textrm{#1}}}}
\newcommand{\subscript}[1]{\ensuremath{_{\textrm{#1}}}}


\newcommand{\chapternumber}{\thechapter}
\renewcommand{\appendixname}{Bijlage}
\renewcommand{\appendixtocname}{Bijlagen}
\renewcommand{\appendixpagename}{Bijlagen}

\usepackage[hidelinks]{hyperref} %<--------ALTIJD ALS LAATSTE

\renewcommand{\familydefault}{\sfdefault}

\setmainfont[Ligatures=TeX]{Myriad Pro}
\setmathfont{Asana Math}
\setmonofont{Lucida Console}

\usepackage{titlesec, blindtext, color}
\definecolor{gray75}{gray}{0.75}
\newcommand{\hsp}{\hspace{20pt}}
\titleformat{\chapter}[hang]{\Huge\bfseries}{\chapternumber\hsp\textcolor{gray75}{|}\hsp}{0pt}{\Huge\bfseries}
\renewcommand{\familydefault}{\sfdefault}
\renewcommand{\arraystretch}{1.2}
\setlength\parindent{0pt}

%For code listings
\definecolor{black}{rgb}{0,0,0}
\definecolor{browntags}{rgb}{0.65,0.1,0.1}
\definecolor{bluestrings}{rgb}{0,0,1}
\definecolor{graycomments}{rgb}{0.4,0.4,0.4}
\definecolor{redkeywords}{rgb}{1,0,0}
\definecolor{bluekeywords}{rgb}{0.13,0.13,0.8}
\definecolor{greencomments}{rgb}{0,0.5,0}
\definecolor{redstrings}{rgb}{0.9,0,0}
\definecolor{purpleidentifiers}{rgb}{0.01,0,0.01}


\lstdefinestyle{csharp}{
language=[Sharp]C,
showspaces=false,
showtabs=false,
breaklines=true,
showstringspaces=false,
breakatwhitespace=true,
escapeinside={(*@}{@*)},
columns=fullflexible,
commentstyle=\color{greencomments},
keywordstyle=\color{bluekeywords}\bfseries,
stringstyle=\color{redstrings},
identifierstyle=\color{purpleidentifiers},
basicstyle=\ttfamily\small}

\lstdefinestyle{c}{
language=C,
showspaces=false,
showtabs=false,
breaklines=true,
showstringspaces=false,
breakatwhitespace=true,
escapeinside={(*@}{@*)},
columns=fullflexible,
commentstyle=\color{greencomments},
keywordstyle=\color{bluekeywords}\bfseries,
stringstyle=\color{redstrings},
identifierstyle=\color{purpleidentifiers},
}

\lstdefinestyle{matlab}{
language=Matlab,
showspaces=false,
showtabs=false,
breaklines=true,
showstringspaces=false,
breakatwhitespace=true,
escapeinside={(*@}{@*)},
columns=fullflexible,
commentstyle=\color{greencomments},
keywordstyle=\color{bluekeywords}\bfseries,
stringstyle=\color{redstrings},
identifierstyle=\color{purpleidentifiers}
}

\lstdefinestyle{vhdl}{
language=VHDL,
showspaces=false,
showtabs=false,
breaklines=true,
showstringspaces=false,
breakatwhitespace=true,
escapeinside={(*@}{@*)},
columns=fullflexible,
commentstyle=\color{greencomments},
keywordstyle=\color{bluekeywords}\bfseries,
stringstyle=\color{redstrings},
identifierstyle=\color{purpleidentifiers}
}

\lstdefinestyle{xaml}{
language=XML,
showspaces=false,
showtabs=false,
breaklines=true,
showstringspaces=false,
breakatwhitespace=true,
escapeinside={(*@}{@*)},
columns=fullflexible,
commentstyle=\color{greencomments},
keywordstyle=\color{redkeywords},
stringstyle=\color{bluestrings},
tagstyle=\color{browntags},
morestring=[b]",
  morecomment=[s]{<?}{?>},
  morekeywords={xmlns,version,typex:AsyncRecords,x:Arguments,x:Boolean,x:Byte,x:Char,x:Class,x:ClassAttributes,x:ClassModifier,x:Code,x:ConnectionId,x:Decimal,x:Double,x:FactoryMethod,x:FieldModifier,x:Int16,x:Int32,x:Int64,x:Key,x:Members,x:Name,x:Object,x:Property,x:Shared,x:Single,x:String,x:Subclass,x:SynchronousMode,x:TimeSpan,x:TypeArguments,x:Uid,x:Uri,x:XData,Grid.Column,Grid.ColumnSpan,Click,ClipToBounds,Content,DropDownOpened,FontSize,Foreground,Header,Height,HorizontalAlignment,HorizontalContentAlignment,IsCancel,IsDefault,IsEnabled,IsSelected,Margin,MinHeight,MinWidth,Padding,SnapsToDevicePixels,Target,TextWrapping,Title,VerticalAlignment,VerticalContentAlignment,Width,WindowStartupLocation,Binding,Mode,OneWay,xmlns:x}
}

%defaults
\lstset{
basicstyle=\ttfamily\small,
extendedchars=false,
numbers=left,
numberstyle=\ttfamily\tiny,
stepnumber=1,
tabsize=4,
numbersep=5pt
}
\addbibresource{../../library/bibliography.bib}

\author{}
\title{EPO3: Eindrapport - Resultaten}

\begin{document}
\chapter{Resultaten}
\label{ch:resultaten}




\subsection{VHDL code}




\subsection{Synthese}




\subsection{FPGA test}






\end{document}


\chapter{Conclusie}
%De conclusie is zeer belangrijk. Deze is meestal relatief kort, en vat heel beknopt de resultaten en
%belangrijkste reflectie daarop nog eens samen. Wanneer de conclusie te lang zou worden, is het beter
%om een aparte discussie sectie (zie hierboven) toe te voegen. In de conclusie kun je, indien van
%toepassing, aanbevelingen doen voor verder werk/onderzoek. (Indien dat te lang zou worden, kun je
%daar ook weer een aparte sectie van maken.) In de conclusie maak je ook eventuele slotopmerkingen.
%Hou er rekening mee dat er lezers zijn die alleen het abstract en de conclusie lezen, en maak deze
%stukken ook voor hen nuttig, bijvoorbeeld om te beslissen of ze het gehele rapport gaan lezen


De probleemstelling bevatte kernachtig het volgende: Bepaal de dikte van het gateoxide van de NMOS transistor. De dikte is bepaald via de capaciteit van het polysilicium. Deze dikte is voor de PMOS 25.0 nanometer en voor de NMOS 25.3 nanometer. Deze diktes zijn bepaald met het computerprogramma SPICE en niet met echte transistors. Een aanbeveling is om ook echte transistors door te meten, als dat mogelijk is. \\ \newline
Hoe kijken we als groepje tegen de simulatie? Eerst hebben we erg lang allerlei circuits geprobeerd, waarna we erachter kwamen dat we de goede grafiek eigenlijk al hadden. Hij was alleen nog niet te beredeneren en dat is nu wel het geval. Deze grafieken vormden de basis voor het uitrekenen van de capaciteiten door het invullen van de functie in SPICE. Verder was het handig om matlab te gebruiken bij het maken van grafieken en bij het afleiden van data uit die grafieken. (basic linear fitting) Daarna was het nog wel even lastig om een gestructureerd verslag in elkaar te zetten, nu was immers geen template beschikbaar. Al met al is de opdracht geslaagd. 



\chapter {Discussie}
\subimport{inputfiles/}{discussie}

\begin{thebibliography}{La}
%Internetbron
%Auteur, A. (jaar van uitgave). Titel van het document. Geraadpleegd op dag maand jaar, van http://url
%
%Hoofdstuk uit boek
%Auteur, A. (jaar van uitgave). Titel van hoofdstuk. In A. Redacteur (Red.), Titel van het boek. (pp. xx-xx). Plaats: %Uitgeverij.
\bibitem{verwijzing}
Rabae. Jan.M(2009). Chapter 3, The Devices, in Rabae, Digital integrated circuits, Pearson education, upper seadle river

\bibitem{verwijzing2}
Laurier, among Canada's best, \textit{PSpice Tutorial|} Geraadpleegd op 14-10-2013, van http://denethor.wlu.ca/PSpice/pspice\_tutorial.html\#vpulse

\bibitem{verwijzing3}
Stack exchange document, \textit{Cascaded logic inverters}, Geraadpleegd op 14-10-2013, van http://electronics.stackexchange.com/questions/30287/cascaded-logic-inverters

\bibitem{verwijzing4}
edaboard.com, overshoot, undershoot capacitor charching, Geraadpleegd op 14-10-2013, van http://www.edaboard.com/thread119364.html

 
\end{thebibliography}

\appendix
\chapter{Matlabcode}
\label{Aa}
\section{Code om grafieken netter te plotten in matlab}
\label{A1}
\tiny
\ci {Klokbuffer.m}
\normalsize
Deze code bevat vectoren om alle data, geëxporteerd uit SPICE, goed te plotten. In de figuur staat slecht een gedeelte van de vector, die loopt nog een eind naar rechts door. Het is dus slechts illustratief. 


\end{document}