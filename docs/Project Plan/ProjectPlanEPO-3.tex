\documentclass{scrartcl} 
% Include all project wide packages here.
\usepackage{fullpage}
\usepackage{polyglossia}
\setmainlanguage{dutch}
\usepackage{csquotes}
\usepackage{graphicx}
\usepackage{epstopdf}
\usepackage{pdfpages}
\usepackage{caption}
\usepackage[list=true]{subcaption}
\usepackage{float}
%\usepackage{mathtools}
\usepackage{standalone}
\usepackage{import}
\usepackage{tocloft}
\usepackage{wrapfig}
\usepackage{authblk}
\usepackage{array}
\usepackage{booktabs}
\usepackage[toc,page,title,titletoc]{appendix}
\usepackage{xunicode}
\usepackage{amsmath}
\usepackage{fontspec}
\usepackage{unicode-math}
\usepackage[
    backend=bibtexu,
	texencoding=utf8,
bibencoding=utf8,
    style=ieee,
    sortlocale=nl_NL,
    language=auto
]{biblatex}
\usepackage{listings}
\newcommand{\includecode}[3][c]{\lstinputlisting[caption=#2, escapechar=, style=#1]{#3}}
\newcommand{\superscript}[1]{\ensuremath{^{\textrm{#1}}}}
\newcommand{\subscript}[1]{\ensuremath{_{\textrm{#1}}}}


\newcommand{\chapternumber}{\thechapter}
\renewcommand{\appendixname}{Bijlage}
\renewcommand{\appendixtocname}{Bijlagen}
\renewcommand{\appendixpagename}{Bijlagen}

\usepackage[hidelinks]{hyperref} %<--------ALTIJD ALS LAATSTE

\renewcommand{\familydefault}{\sfdefault}

\setmainfont[Ligatures=TeX]{Myriad Pro}
\setmathfont{Asana Math}
\setmonofont{Lucida Console}

\usepackage{titlesec, blindtext, color}
\definecolor{gray75}{gray}{0.75}
\newcommand{\hsp}{\hspace{20pt}}
\titleformat{\chapter}[hang]{\Huge\bfseries}{\chapternumber\hsp\textcolor{gray75}{|}\hsp}{0pt}{\Huge\bfseries}
\renewcommand{\familydefault}{\sfdefault}
\renewcommand{\arraystretch}{1.2}
\setlength\parindent{0pt}

%For code listings
\definecolor{black}{rgb}{0,0,0}
\definecolor{browntags}{rgb}{0.65,0.1,0.1}
\definecolor{bluestrings}{rgb}{0,0,1}
\definecolor{graycomments}{rgb}{0.4,0.4,0.4}
\definecolor{redkeywords}{rgb}{1,0,0}
\definecolor{bluekeywords}{rgb}{0.13,0.13,0.8}
\definecolor{greencomments}{rgb}{0,0.5,0}
\definecolor{redstrings}{rgb}{0.9,0,0}
\definecolor{purpleidentifiers}{rgb}{0.01,0,0.01}


\lstdefinestyle{csharp}{
language=[Sharp]C,
showspaces=false,
showtabs=false,
breaklines=true,
showstringspaces=false,
breakatwhitespace=true,
escapeinside={(*@}{@*)},
columns=fullflexible,
commentstyle=\color{greencomments},
keywordstyle=\color{bluekeywords}\bfseries,
stringstyle=\color{redstrings},
identifierstyle=\color{purpleidentifiers},
basicstyle=\ttfamily\small}

\lstdefinestyle{c}{
language=C,
showspaces=false,
showtabs=false,
breaklines=true,
showstringspaces=false,
breakatwhitespace=true,
escapeinside={(*@}{@*)},
columns=fullflexible,
commentstyle=\color{greencomments},
keywordstyle=\color{bluekeywords}\bfseries,
stringstyle=\color{bluestrings},
identifierstyle=\color{purpleidentifiers}
}

\lstdefinestyle{vhdl}{
language=VHDL,
showspaces=false,
showtabs=false,
breaklines=true,
showstringspaces=false,
breakatwhitespace=true,
escapeinside={(*@}{@*)},
columns=fullflexible,
commentstyle=\color{greencomments},
keywordstyle=\color{bluekeywords}\bfseries,
stringstyle=\color{redstrings},
identifierstyle=\color{purpleidentifiers}
}

\lstdefinestyle{xaml}{
language=XML,
showspaces=false,
showtabs=false,
breaklines=true,
showstringspaces=false,
breakatwhitespace=true,
escapeinside={(*@}{@*)},
columns=fullflexible,
commentstyle=\color{greencomments},
keywordstyle=\color{redkeywords},
stringstyle=\color{bluestrings},
tagstyle=\color{browntags},
morestring=[b]",
  morecomment=[s]{<?}{?>},
  morekeywords={xmlns,version,typex:AsyncRecords,x:Arguments,x:Boolean,x:Byte,x:Char,x:Class,x:ClassAttributes,x:ClassModifier,x:Code,x:ConnectionId,x:Decimal,x:Double,x:FactoryMethod,x:FieldModifier,x:Int16,x:Int32,x:Int64,x:Key,x:Members,x:Name,x:Object,x:Property,x:Shared,x:Single,x:String,x:Subclass,x:SynchronousMode,x:TimeSpan,x:TypeArguments,x:Uid,x:Uri,x:XData,Grid.Column,Grid.ColumnSpan,Click,ClipToBounds,Content,DropDownOpened,FontSize,Foreground,Header,Height,HorizontalAlignment,HorizontalContentAlignment,IsCancel,IsDefault,IsEnabled,IsSelected,Margin,MinHeight,MinWidth,Padding,SnapsToDevicePixels,Target,TextWrapping,Title,VerticalAlignment,VerticalContentAlignment,Width,WindowStartupLocation,Binding,Mode,OneWay,xmlns:x}
}

%defaults
\lstset{
basicstyle=\ttfamily\small,
extendedchars=false,
numbers=left,
numberstyle=\ttfamily\tiny,
stepnumber=1,
tabsize=4,
numbersep=5pt
}
\addbibresource{../library/bibliography.bib}

\title{EPO-3 "Maak een Chip"}
\author{Groep 1}
\date{}

\begin{document}
\maketitle
\pagenumbering{roman}
\vspace{80 mm}
\section*{Samenvatting}
\newpage
\setlength{\cftbeforetoctitleskip}{-3em}

\tableofcontents
\clearpage

\section{Achtergrond}
Het project "Maak een Chip"\ is een vervolg op de projecten uit het eerste jaar van de bachelor Electrical Engineering, waarbij de nadruk is gelegd op het gestructureerd hierarchisch ontwerpen van een chip en zoals bij elk project het samenwerken in een groep. Dit wordt gedaan met een groep van 9 bsc electrical engineering studenten. Wanneer het ontwerp af is, wordt deze in het 2e semester tot werkelijkheid gebracht in de chipfabriek DIMES en in het 4de kwartaal zal de chip af zijn en getest kunnen worden. Bij dit project zal de kennis van bepaalde vakken zoals lineaire schakelingen, halfgeleiders en versterkerschakelingen uit het eerste jaar electrical engineering toegepast worden om zo een goed en werkend product te krijgen.
\section{Projectopdracht}
Het uiteindelijke resultaat dat moet worden bereikt met dit project is het maken van een correct ontwerp van een chip met alle bijbehorende stappen. Om dit te kunnen bereiken dienen een aantal doelen bereikt te worden;
\begin {itemize}
\item Het ontwerpen aan de hand van algemene producspecificaties met randvoorwaarden.
\end {itemize}
\quad Het is belangrijk voor het ontwerpen van de chip, dat de productspecificaties met randvoorwaarden zo goed en duidelijk mogelijk worden gespecificeerd. 
Zo zal het ontwerpen van de chip makkelijker verlopen en bepaalde problemen en limitaties zullen voor de deadline verholpen kunnen worden.
\begin {itemize}
\item Het ontwerp systematisch en hierarchisch opstellen.
\end {itemize}
\quad Door het ontwerp systematisch en hierarchisch te maken hou je een duidelijke structuur in het geheel. Aanpassingen zoals het opzoeken en repareren van fouten, het toevoegen van modules en het aanpassen van modules zullen door deze werkwijze veel gemakkelijker zijn.
\begin {itemize}
\item Het ontwerp analyseren 
\end {itemize}
\quad Het is belangrijk om het ontwerp op verschillende niveau's te analyseren met behulp van simulaties, prototyping en metingen. Op deze manier kan worden getest of het ontwerp ook op switch-level en transistor niveau mogelijk is. Wanneer dit niet zo is kan het ontwerp worden aangepast zodoende dat dit wel mogelijk is.
\begin {itemize}
\item Het ontwerp opstellen aan de hand van verscheidene modellen
\end {itemize}
\quad Bij het modelleren van het ontwerp is het belangrijk om de beperkingen van het gebruikte model in de gaten te houden, zodat er rekening kan worden gehouden met uitkomsten die het model niet kan verklaren. Om deze reden is het handig met behulp van verschillende modellen te modeleren zodat hierover een duidelijk beeld onstaat.
\begin {itemize}
\item Het ontwerp testbaar maken
\end {itemize}
\quad Bij het ontwerpen van de chip moet ook op de testbaarheid van het ontwerp gelet worden. Wanneer dit niet gebeurd zal het ontwerp niet goed te simuleren zijn en zal het moeilijk zijn om fouten en beperkingen te vinden in het ontwerp.

\section{Projectactiviteiten}


Wat zijn de verschillende activiteiten die moeten worden uitgevoerd om het eindresultaat te bereiken ?   Probeer - mede aan de hand van de ervaring met de FSM en de moduleopdracht - zo goed mogelijk in te schatten welke stappen hiervoor moeten worden gezet.

\section{Specificaties}
\subsection {Functionele Eisen}
Blablabla
\subsection {Randvoorwaarden}
\begin {itemize}
\item De algemene technische randvoorwaarden zijn bepaald in hoofdstuk 11.8 "\Algemene Technologische Randvoorwaarden"  van het EPO-3 Handleiding/cite{epo3-manual}
\item Beeldresolutie van 640x480 in 16 kleuren.
\end{itemize}
\section{Producten}
\begin{itemize}
\item Het Projectplan
\item Specificaties \& Randvoorwaarden van GPU
\item Functieblokschema Ontwerp GPU
\item Ontwerp IC
\item FPGA Prototype
\item Het Tussenrapport
\item Het Chip Ontwerp
\item Bestanden van de volledige Chip
\item Het Eindverslag
\end {itemize}

\section{Kwaliteit}
Alle ontwerpbeslissingen worden met minimaal 50\% van de groep besproken.
\\Het resultaat van een bepaalde module wordt na uitwerking besproken met de groep om zekerheid te verkrijgen van zijn functionaliteit.
\\Alle modules moeten in VHDL geschreven worden en er moet rekening gehouden worden met de VHDL restricties. Alle modules worden gesimuleerd vervolgens gesynthetiseerd en tot slot \\wordt er place en route gebruikt om zekerheid te verkrijgen dat de chip op alle niveau's, namelijk VHDL, switch level en transistor, volledig functioneel is.
\\De chip moet geprototyped worden op de FPGA chip om zo er zeker van te zijn dat de chip in theorie werkt.

\section{Projectorganisatie}
\begin {tabular}{|c|c|}
\hline
Marian Bartek & Tutor\\
\hline
Joris Coenen & Studentmentor\\
\hline
Efraïm Eland & Projectlid\\
\hline
Robin Hes & Voorzitter\\
\hline
Tu Hoang & Projectlid\\
\hline
Kees Hogenhout & Projectlid\\
\hline
Alex Janssen & Projectlid\\
\hline
Jorden Kerkhof & Projectlid\\
\hline
Peter Stijnman & Projectlid\\
\hline
Xenia Wesdijk & Projectlid\\
\hline
Erwin de Haan & Projectlid\\
\hline
\end {tabular}
\\
\\Er zullen elke middag twee vergaderingen plaatsvinden. Een openingsvergadering om 8:50 en een afsluitende vergadering om 12:30.
Alle documentatie zal geplaatst worden op github met dropbox als backup.
Alle offici"ele documentatie zal gemaakt worden in Latex en worden gepubliceerd in pdf formaat.

\section{Planning}

Als je weet welke activiteiten er worden uitgevoerd, welke producten (milestones) er worden geleverd, en wie deelnemen aan het project, dan kan er een planning worden gemaakt. Hiervoor kan eventueel software zoals MicroSoft Project of GanttProject gebruikt worden.  De volgende overzichten zijn bijvoorbeeld gegenereerd m.b.v. het gratis te downloaden pakket GanttProject.



\section{Risico's}
Wat zijn de gevaren die het succes het van het project bedreigen ? Wat doe je om de risico's op falen zo klein mogelijk te houden? Een van de mogelijke maatregelen is het werken met 'fall-back' scenario’s. Welke scenario's zijn dit dan?


\end{document}

