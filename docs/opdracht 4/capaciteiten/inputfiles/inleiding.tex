
In de hedendaagse technologie wordt veelvuldig gebruikt gemaakt van transistoren, met name de NMOS en de PMOS transistoren. Deze transistoren kennen veel voordelen en worden nog steeds elke dag geoptimaliseerd. Om beter inzicht te krijgen in de werking van deze transistoren en eventueel verbeteringen te kunnen maken, moet er onderzoek worden gedaan naar het gedrag van deze transistoren. Een van de eigenschappen van een NMOS en PMOS transistor, waar zorgvuldig rekening mee moet worden gehouden in schakelingen, is de parasitaire capaciteit. Met deze parasitaire capaciteit kunnen veel eigenschappen van de transistor worden berekend. In dit verslag wordt de dikte van het oxide in de transistor bepaald aan de hand van deze parasitaire capaciteit. Dit verslag heeft niet als doel de transistor te verbeteren, maar alleen inzicht te krijgen in de structuur en werking van capaciteiten in een MOS transistor.
\\ \newline
In dit verslag wordt eerst theorie behandeld om tot de berekeningen te komen. Vervolgens worden de eisen behandeld, waarna een methode wordt vastgesteld. Daarna zijn de resultaten van die methode te zien. Als laatst wordt een discussie en een conclusie over deze opdracht geschreven. 


