%In de ‘discussie’ sectie reflecteer je op de resultaten. Een uitkomst zonder discussie en/of reflectie
%heeft zelden ‘impact’. Welke doelen zijn wel bereikt, welke niet. Komt je resultaat overeen met de
%verwachte uitkomst? Zo niet, wat zou de oorzaak van het verschil kunnen zijn? Wat zijn de sterkepunten van je werk, en %welke de zwakke? Wat zou je een andere keer anders doen? Bespreek, indien
%van toepassing, implicaties, van je werk. Voorbeeld: wat is het mogelijke gevolg van Neutrino’s die
%sneller reizen dan het licht? Wat zijn mogelijke bronnen van meetfouten of interpretatiefouten?

In het vorige stukje, bij analyse, is al ingegaan op de redelijkheid van de resultaten. Een zwak punt van dit onzerzoek is dat het gebaseerd is op simulaties in SPICE. Dat betekent dat je nu alleen de werkelijkheid nabootst, wellicht met te ideale omstandigheden. Om het echt goed te doen, zou je echt een NMOS en een PMOS aan moeten sluiten dat in SPICE ook gedaan is. Dan heb je ook direct een check hoe goed je simulaties zijn. Verder is nu een standaard transistor gekozen in SPICE, terwijl  een transistor in werkelijkheid altijd toleranties heeft, die een fabrikant opgeeft. Deze toleranties ontstaan door het fabricageproces. (bijv 5\% afwijking van de standaardstroom die hij kan leveren). \\
Ook zijn nu formules gebruikt die wellicht niet alle randeffecten ondervangen. Dat geeft ook een afwijking in je capaciteit. Je zou het onderzoek kunnen uitbreiden door ook de andere capaciteiten te berekenen. Nu wordt namelijk alleen de \textit{C\tss{GC}} berekend, terwijl meer capaciteiten een rol spelen, zoals geïntroduceerd bij geïntegreerde schakelingen. Uiteindelijk zijn de doelen bereikt en zijn er logisch te beredeneren antwoorden uitgekomen. 

