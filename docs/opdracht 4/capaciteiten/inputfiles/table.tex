


\begin{table} [h!]
\centering % 
\begin{tabular}{l c c c c c} 
\toprule % Top horizontal line
& \multicolumn {4}{c}{Formula variables} \\ % Amalgamating several columns into one cell is done using the \multicolumn command as seen on this line
\cmidrule(l){1-3} % Horizontal line spanning less than the full width of the table - you can add (r) or (l) just before the opening curly bracket to shorten the rule on the left or right side
L($\mu$m)&  W($\mu$m)  & C\tss{g}\\ % Column names row
\midrule % In-table horizontal line
2.2 & 3.0  & 8.3381 \\ % Content row 1
2.4 & 3.0 & 9.1672 \\ % Content row 2
2.8 & 3.0 & 10.825 \\ % Content row 3
3.2 & 3.0 & 12.483 \\ % Content row 4
3.6 & 3.0 & 14.141 \\ % Content row 5
4.0 & 3.0 & 15.799 \\ % Content row 6
4.4 & 3.0 & 17.457 \\ % Content row 7
4.8 & 3.0 & 19.115 \\ % Content row 8
\midrule % In-table horizontal line
\midrule % In-table horizontal line
%Average Rate & 0.920 & 0.882 \\ % Summary/total row
\bottomrule % Bottom horizontal line
\end{tabular}
\caption{Table caption text} % Table caption, can be commented out if no caption is required
\label{tab:template} % A label for referencing this table elsewhere, references are used in text as \ref{label}
\end{table}

