

Bepaal door middel van simulaties aan een transitor de dikte van het gate oxide. Daarvoor moet je eerst de gate-channel capaciteit per oppervlakte-eenheid bepalen. Hou er rekening mee dat de totale capaciteit ook nog een bijdrage kent van de zogenaamde \emph{lateral diffusion}. De totale capaciteit \emph{C\tss{G}} wordt dan beschreven door de volgende formule, met \emph{W} de breedte, \emph{L} de lengte van de transistor, \emph{C\tss{o}} de zogenaamde overlap capaciteit en \emph{C\tss{GC}} de gezochte kanaalcapaciteit:
\begin{equation}
\emph{C\tss{G} = W \ast C\tss{o} + L \ast W \ast C\tss{GC}}
\end{equation}
Door transistoren met verschillende afmetingen te nemen kun je de termen \emph{C\tss{o}} en \emph{C\tss{GC}} apart bepalen. Wanneer je \emph{C\tss{GC}} hebt, kun je met behulp van de permittiviteit van het gate materiaal (silicium dioxide, \emph{SiO\tss{2}}) de gezochte oxide dikte bepalen. In voorbeeld 3.9 van het boek van Rabaey [1] wordt uitgelegd hoe je de totale gate capaciteit kunt bepalen door middel van simulatie. Hou er rekening mee dat je alleen op zoek bent naar het vlakke deel van de grafiek in figuur 3.32 van Rabaey [1]. Dit vlakke deel is gelijk aan de waarde \emph{WLC\tss{ox}} in figuur 3.31 van Rabaey [1]. Doe de simulaties en berekeningen voor zowel de NMOS transistor als de PMOS transistor en bespreek de resultaten.



