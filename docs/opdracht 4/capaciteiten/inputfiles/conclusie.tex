%De conclusie is zeer belangrijk. Deze is meestal relatief kort, en vat heel beknopt de resultaten en
%belangrijkste reflectie daarop nog eens samen. Wanneer de conclusie te lang zou worden, is het beter
%om een aparte discussie sectie (zie hierboven) toe te voegen. In de conclusie kun je, indien van
%toepassing, aanbevelingen doen voor verder werk/onderzoek. (Indien dat te lang zou worden, kun je
%daar ook weer een aparte sectie van maken.) In de conclusie maak je ook eventuele slotopmerkingen.
%Hou er rekening mee dat er lezers zijn die alleen het abstract en de conclusie lezen, en maak deze
%stukken ook voor hen nuttig, bijvoorbeeld om te beslissen of ze het gehele rapport gaan lezen


De probleemstelling bevatte kernachtig het volgende: Bepaal de dikte van het gateoxide van de NMOS transistor. De dikte is bepaald via de capaciteit van het polysilicium. Deze dikte is voor de PMOS 25.0 nanometer en voor de NMOS 25.3 nanometer. Deze diktes zijn bepaald met het computerprogramma SPICE en niet met echte transistors. Een aanbeveling is om ook echte transistors door te meten, als dat mogelijk is. \\ \newline
Hoe kijken we als groepje tegen de simulatie? Eerst hebben we erg lang allerlei circuits geprobeerd, waarna we erachter kwamen dat we de goede grafiek eigenlijk al hadden. Hij was alleen nog niet te beredeneren en dat is nu wel het geval. Deze grafieken vormden de basis voor het uitrekenen van de capaciteiten door het invullen van de functie in SPICE. Verder was het handig om matlab te gebruiken bij het maken van grafieken en bij het afleiden van data uit die grafieken. (basic linear fitting) Daarna was het nog wel even lastig om een gestructureerd verslag in elkaar te zetten, nu was immers geen template beschikbaar. Al met al is de opdracht geslaagd. 

