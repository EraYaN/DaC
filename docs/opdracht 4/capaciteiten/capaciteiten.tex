\documentclass{report}
% Include all project wide packages here.
\usepackage{fullpage}
\usepackage{polyglossia}
\setmainlanguage{dutch}
\usepackage{csquotes}
\usepackage{graphicx}
\usepackage{epstopdf}
\usepackage{pdfpages}
\usepackage{caption}
\usepackage[list=true]{subcaption}
\usepackage{float}
\usepackage{standalone}
\usepackage{import}
\usepackage{tocloft}
\usepackage{wrapfig}
\usepackage{authblk}
\usepackage{array}
\usepackage{booktabs}
\usepackage[toc,page,title,titletoc]{appendix}
\usepackage{xunicode}
\usepackage{fontspec}
\usepackage{pgfplots}
\usepackage{SIunits}
\pgfplotsset{compat=newest}
\pgfplotsset{plot coordinates/math parser=false}
\newlength\figureheight 
\newlength\figurewidth
\usepackage{unicode-math}
\usepackage[
    backend=bibtexu,
	texencoding=utf8,
bibencoding=utf8,
    style=ieee,
    sortlocale=nl_NL,
    language=auto
]{biblatex}
\usepackage{listings}
\newcommand{\includecode}[3][c]{\lstinputlisting[caption=#2, escapechar=, style=#1]{#3}}
\newcommand{\superscript}[1]{\ensuremath{^{\textrm{#1}}}}
\newcommand{\subscript}[1]{\ensuremath{_{\textrm{#1}}}}


\newcommand{\chapternumber}{\thechapter}
\renewcommand{\appendixname}{Bijlage}
\renewcommand{\appendixtocname}{Bijlagen}
\renewcommand{\appendixpagename}{Bijlagen}

\usepackage[hidelinks]{hyperref} %<--------ALTIJD ALS LAATSTE

%\renewcommand{\familydefault}{\sfdefault}

\setmainfont[Ligatures=TeX]{Myriad Pro}
\setmathfont{Asana Math}
\setmonofont{Lucida Console}

\usepackage{titlesec, blindtext, color}
\definecolor{gray75}{gray}{0.75}
\newcommand{\hsp}{\hspace{20pt}}
\titleformat{\chapter}[hang]{\Huge\bfseries}{\chapternumber\hsp\textcolor{gray75}{|}\hsp}{0pt}{\Huge\bfseries}
\renewcommand{\familydefault}{\sfdefault}
\renewcommand{\arraystretch}{1.2}
\setlength\parindent{0pt}

%For code listings
\definecolor{black}{rgb}{0,0,0}
\definecolor{browntags}{rgb}{0.65,0.1,0.1}
\definecolor{bluestrings}{rgb}{0,0,1}
\definecolor{graycomments}{rgb}{0.4,0.4,0.4}
\definecolor{redkeywords}{rgb}{1,0,0}
\definecolor{bluekeywords}{rgb}{0.13,0.13,0.8}
\definecolor{greencomments}{rgb}{0,0.5,0}
\definecolor{redstrings}{rgb}{0.9,0,0}
\definecolor{purpleidentifiers}{rgb}{0.01,0,0.01}


\lstdefinestyle{csharp}{
language=[Sharp]C,
showspaces=false,
showtabs=false,
breaklines=true,
showstringspaces=false,
breakatwhitespace=true,
escapeinside={(*@}{@*)},
columns=fullflexible,
commentstyle=\color{greencomments},
keywordstyle=\color{bluekeywords}\bfseries,
stringstyle=\color{redstrings},
identifierstyle=\color{purpleidentifiers},
basicstyle=\ttfamily\small}

\lstdefinestyle{c}{
language=C,
showspaces=false,
showtabs=false,
breaklines=true,
showstringspaces=false,
breakatwhitespace=true,
escapeinside={(*@}{@*)},
columns=fullflexible,
commentstyle=\color{greencomments},
keywordstyle=\color{bluekeywords}\bfseries,
stringstyle=\color{redstrings},
identifierstyle=\color{purpleidentifiers},
}

\lstdefinestyle{matlab}{
language=Matlab,
showspaces=false,
showtabs=false,
breaklines=true,
showstringspaces=false,
breakatwhitespace=true,
escapeinside={(*@}{@*)},
columns=fullflexible,
commentstyle=\color{greencomments},
keywordstyle=\color{bluekeywords}\bfseries,
stringstyle=\color{redstrings},
identifierstyle=\color{purpleidentifiers}
}

\lstdefinestyle{vhdl}{
language=VHDL,
showspaces=false,
showtabs=false,
breaklines=true,
showstringspaces=false,
breakatwhitespace=true,
escapeinside={(*@}{@*)},
columns=fullflexible,
commentstyle=\color{greencomments},
keywordstyle=\color{bluekeywords}\bfseries,
stringstyle=\color{redstrings},
identifierstyle=\color{purpleidentifiers}
}

\lstdefinestyle{xaml}{
language=XML,
showspaces=false,
showtabs=false,
breaklines=true,
showstringspaces=false,
breakatwhitespace=true,
escapeinside={(*@}{@*)},
columns=fullflexible,
commentstyle=\color{greencomments},
keywordstyle=\color{redkeywords},
stringstyle=\color{bluestrings},
tagstyle=\color{browntags},
morestring=[b]",
  morecomment=[s]{<?}{?>},
  morekeywords={xmlns,version,typex:AsyncRecords,x:Arguments,x:Boolean,x:Byte,x:Char,x:Class,x:ClassAttributes,x:ClassModifier,x:Code,x:ConnectionId,x:Decimal,x:Double,x:FactoryMethod,x:FieldModifier,x:Int16,x:Int32,x:Int64,x:Key,x:Members,x:Name,x:Object,x:Property,x:Shared,x:Single,x:String,x:Subclass,x:SynchronousMode,x:TimeSpan,x:TypeArguments,x:Uid,x:Uri,x:XData,Grid.Column,Grid.ColumnSpan,Click,ClipToBounds,Content,DropDownOpened,FontSize,Foreground,Header,Height,HorizontalAlignment,HorizontalContentAlignment,IsCancel,IsDefault,IsEnabled,IsSelected,Margin,MinHeight,MinWidth,Padding,SnapsToDevicePixels,Target,TextWrapping,Title,VerticalAlignment,VerticalContentAlignment,Width,WindowStartupLocation,Binding,Mode,OneWay,xmlns:x}
}

%defaults
\lstset{
basicstyle=\ttfamily\small,
extendedchars=false,
numbers=left,
numberstyle=\ttfamily\tiny,
stepnumber=1,
tabsize=4,
numbersep=5pt
}

\begin{document}
\newcommand{\rp}{$\rightarrow$}
\newcommand{\Ohm}{$\Omega$}
\newcommand{\ohm}{$\omega$}
\newcommand{\gmu}{$\mu$}
\newcommand{\tss}{\textsubscript}
\newcommand{\lijst}{\lstinputlisting}

%%%%%%%%%%%%%%%%%%%%%%%%%%%%%%%%%%%%%%%%%
% University Assignment Title Page 
% LaTeX Template
% Version 1.0 (27/12/12)
%
% This template has been downloaded from:
% http://www.LaTeXTemplates.com
%
% Original author:
% WikiBooks (http://en.wikibooks.org/wiki/LaTeX/Title_Creation)
%
% License:
% CC BY-NC-SA 3.0 (http://creativecommons.org/licenses/by-nc-sa/3.0/)
% 
% Instructions for using this template:
% This title page is capable of being compiled as is. This is not useful for 
% including it in another document. To do this, you have two options: 
%
% 1) Copy/paste everything between \begin{document} and \end{document} 
% starting at \begin{titlepage} and paste this into another LaTeX file where you 
% want your title page.
% OR
% 2) Remove everything outside the \begin{titlepage} and \end{titlepage} and 
% move this file to the same directory as the LaTeX file you wish to add it to. 
% Then add %%%%%%%%%%%%%%%%%%%%%%%%%%%%%%%%%%%%%%%%%
% University Assignment Title Page 
% LaTeX Template
% Version 1.0 (27/12/12)
%
% This template has been downloaded from:
% http://www.LaTeXTemplates.com
%
% Original author:
% WikiBooks (http://en.wikibooks.org/wiki/LaTeX/Title_Creation)
%
% License:
% CC BY-NC-SA 3.0 (http://creativecommons.org/licenses/by-nc-sa/3.0/)
% 
% Instructions for using this template:
% This title page is capable of being compiled as is. This is not useful for 
% including it in another document. To do this, you have two options: 
%
% 1) Copy/paste everything between \begin{document} and \end{document} 
% starting at \begin{titlepage} and paste this into another LaTeX file where you 
% want your title page.
% OR
% 2) Remove everything outside the \begin{titlepage} and \end{titlepage} and 
% move this file to the same directory as the LaTeX file you wish to add it to. 
% Then add %%%%%%%%%%%%%%%%%%%%%%%%%%%%%%%%%%%%%%%%%
% University Assignment Title Page 
% LaTeX Template
% Version 1.0 (27/12/12)
%
% This template has been downloaded from:
% http://www.LaTeXTemplates.com
%
% Original author:
% WikiBooks (http://en.wikibooks.org/wiki/LaTeX/Title_Creation)
%
% License:
% CC BY-NC-SA 3.0 (http://creativecommons.org/licenses/by-nc-sa/3.0/)
% 
% Instructions for using this template:
% This title page is capable of being compiled as is. This is not useful for 
% including it in another document. To do this, you have two options: 
%
% 1) Copy/paste everything between \begin{document} and \end{document} 
% starting at \begin{titlepage} and paste this into another LaTeX file where you 
% want your title page.
% OR
% 2) Remove everything outside the \begin{titlepage} and \end{titlepage} and 
% move this file to the same directory as the LaTeX file you wish to add it to. 
% Then add \input{./title_page_1.tex} to your LaTeX file where you want your
% title page.
%
%%%%%%%%%%%%%%%%%%%%%%%%%%%%%%%%%%%%%%%%%

%----------------------------------------------------------------------------------------
%	PACKAGES AND OTHER DOCUMENT CONFIGURATIONS
%----------------------------------------------------------------------------------------


\documentclass[12pt]{article}
\newcommand{\tss}{\textsubscript}
\input{../../../library/preamble}

\begin{document}


\begin{titlepage}

\newcommand{\HRule}{\rule{\linewidth}{0.5mm}} % Defines a new command for the horizontal lines, change thickness here

\center % Center everything on the page
 
%----------------------------------------------------------------------------------------
%	HEADING SECTIONS
%----------------------------------------------------------------------------------------

\textsc{\LARGE TU Delft}\\[1.5cm] % Name of your university/college
\textsc{\Large EE28 EPO-3 Moduleopdracht 5}\\[0.5cm] % Major heading such as course name
\textsc{\large Klokbuffer}\\[0.5cm] % Minor heading such as course title

%----------------------------------------------------------------------------------------
%	TITLE SECTION
%----------------------------------------------------------------------------------------

\HRule \\[0.4cm]
{ \huge \bfseries Klokbuffer  noodzakelijk voor kloknetwerk?}\\[0.4cm] % Title of your document
\HRule \\[1.5cm]
 
%----------------------------------------------------------------------------------------
%	AUTHOR SECTION
%----------------------------------------------------------------------------------------

\begin{minipage}{0.4\textwidth}
\begin{flushleft} \large
\emph{Author:}\\
Kees \textsc{Hogenhout} 4220927\\ % Your name 
Tu \textsc{Hoang} 4203496 %other name
\end{flushleft}
\end{minipage}
~
\begin{minipage}{0.4\textwidth}
\begin{flushright} \large
\emph{Supervisor:} \\
Dr. Marjam \textsc{Bartek} % Supervisor's Name
\end{flushright}
\end{minipage}\\[4cm]

% If you don't want a supervisor, uncomment the two lines below and remove the section above
%\Large \emph{Author:}\\
%John \textsc{Smith}\\[3cm] % Your name

%----------------------------------------------------------------------------------------
%	DATE SECTION
%----------------------------------------------------------------------------------------

{\large \today}\\[3cm] % Date, change the \today to a set date if you want to be precise

%----------------------------------------------------------------------------------------
%	LOGO SECTION
%----------------------------------------------------------------------------------------

%\includegraphics{Logo}\\[1cm] % Include a department/university logo - this will require the graphicx package
 
%----------------------------------------------------------------------------------------
% 

\textbf{Abstract} \\
In dit onderzoekje wordt uitgezocht in welke mate een klokbuffer nodig is voor een netwerk van 125 D-flipflops. Eerst wordt de t-90\% van een inverter bepaald met een C\tss{load} van 25pF (belastingscapaciteit van die 125 DFF's). Vervolgens worden er drie inverters (die groter zijn)  in cascade geplaatst met dezelfde belastingscapaciteit en wordt weer de t-90\% bepaald. Te zien is aan de simulatieresultaten dat de drie inverters als buffer werken. De afvlakking bij een 0-1 overgang van de uitgang is namelijk heel klein, terwijl die afvlakking bij de meting met een enkele transistor veel groter is. 

\vfill % Fill the rest of the page with whitespace

\end{titlepage}
\end{document} to your LaTeX file where you want your
% title page.
%
%%%%%%%%%%%%%%%%%%%%%%%%%%%%%%%%%%%%%%%%%

%----------------------------------------------------------------------------------------
%	PACKAGES AND OTHER DOCUMENT CONFIGURATIONS
%----------------------------------------------------------------------------------------


\documentclass[12pt]{article}
\newcommand{\tss}{\textsubscript}
% Include all project wide packages here.
\usepackage{fullpage}
\usepackage{polyglossia}
\setmainlanguage{dutch}
\usepackage{csquotes}
\usepackage{graphicx}
\usepackage{epstopdf}
\usepackage{pdfpages}
\usepackage{caption}
\usepackage[list=true]{subcaption}
\usepackage{float}
\usepackage{standalone}
\usepackage{import}
\usepackage{tocloft}
\usepackage{wrapfig}
\usepackage{authblk}
\usepackage{array}
\usepackage{booktabs}
\usepackage[toc,page,title,titletoc]{appendix}
\usepackage{xunicode}
\usepackage{fontspec}
\usepackage{pgfplots}
\usepackage{SIunits}
\pgfplotsset{compat=newest}
\pgfplotsset{plot coordinates/math parser=false}
\newlength\figureheight 
\newlength\figurewidth
\usepackage{unicode-math}
\usepackage[
    backend=bibtexu,
	texencoding=utf8,
bibencoding=utf8,
    style=ieee,
    sortlocale=nl_NL,
    language=auto
]{biblatex}
\usepackage{listings}
\newcommand{\includecode}[3][c]{\lstinputlisting[caption=#2, escapechar=, style=#1]{#3}}
\newcommand{\superscript}[1]{\ensuremath{^{\textrm{#1}}}}
\newcommand{\subscript}[1]{\ensuremath{_{\textrm{#1}}}}


\newcommand{\chapternumber}{\thechapter}
\renewcommand{\appendixname}{Bijlage}
\renewcommand{\appendixtocname}{Bijlagen}
\renewcommand{\appendixpagename}{Bijlagen}

\usepackage[hidelinks]{hyperref} %<--------ALTIJD ALS LAATSTE


\begin{document}


\begin{titlepage}

\newcommand{\HRule}{\rule{\linewidth}{0.5mm}} % Defines a new command for the horizontal lines, change thickness here

\center % Center everything on the page
 
%----------------------------------------------------------------------------------------
%	HEADING SECTIONS
%----------------------------------------------------------------------------------------

\textsc{\LARGE TU Delft}\\[1.5cm] % Name of your university/college
\textsc{\Large EE28 EPO-3 Moduleopdracht 5}\\[0.5cm] % Major heading such as course name
\textsc{\large Klokbuffer}\\[0.5cm] % Minor heading such as course title

%----------------------------------------------------------------------------------------
%	TITLE SECTION
%----------------------------------------------------------------------------------------

\HRule \\[0.4cm]
{ \huge \bfseries Klokbuffer  noodzakelijk voor kloknetwerk?}\\[0.4cm] % Title of your document
\HRule \\[1.5cm]
 
%----------------------------------------------------------------------------------------
%	AUTHOR SECTION
%----------------------------------------------------------------------------------------

\begin{minipage}{0.4\textwidth}
\begin{flushleft} \large
\emph{Author:}\\
Kees \textsc{Hogenhout} 4220927\\ % Your name 
Tu \textsc{Hoang} 4203496 %other name
\end{flushleft}
\end{minipage}
~
\begin{minipage}{0.4\textwidth}
\begin{flushright} \large
\emph{Supervisor:} \\
Dr. Marjam \textsc{Bartek} % Supervisor's Name
\end{flushright}
\end{minipage}\\[4cm]

% If you don't want a supervisor, uncomment the two lines below and remove the section above
%\Large \emph{Author:}\\
%John \textsc{Smith}\\[3cm] % Your name

%----------------------------------------------------------------------------------------
%	DATE SECTION
%----------------------------------------------------------------------------------------

{\large \today}\\[3cm] % Date, change the \today to a set date if you want to be precise

%----------------------------------------------------------------------------------------
%	LOGO SECTION
%----------------------------------------------------------------------------------------

%\includegraphics{Logo}\\[1cm] % Include a department/university logo - this will require the graphicx package
 
%----------------------------------------------------------------------------------------
% 

\textbf{Abstract} \\
In dit onderzoekje wordt uitgezocht in welke mate een klokbuffer nodig is voor een netwerk van 125 D-flipflops. Eerst wordt de t-90\% van een inverter bepaald met een C\tss{load} van 25pF (belastingscapaciteit van die 125 DFF's). Vervolgens worden er drie inverters (die groter zijn)  in cascade geplaatst met dezelfde belastingscapaciteit en wordt weer de t-90\% bepaald. Te zien is aan de simulatieresultaten dat de drie inverters als buffer werken. De afvlakking bij een 0-1 overgang van de uitgang is namelijk heel klein, terwijl die afvlakking bij de meting met een enkele transistor veel groter is. 

\vfill % Fill the rest of the page with whitespace

\end{titlepage}
\end{document} to your LaTeX file where you want your
% title page.
%
%%%%%%%%%%%%%%%%%%%%%%%%%%%%%%%%%%%%%%%%%

%----------------------------------------------------------------------------------------
%	PACKAGES AND OTHER DOCUMENT CONFIGURATIONS
%----------------------------------------------------------------------------------------


\documentclass[12pt]{article}
\newcommand{\tss}{\textsubscript}
% Include all project wide packages here.
\usepackage{fullpage}
\usepackage{polyglossia}
\setmainlanguage{dutch}
\usepackage{csquotes}
\usepackage{graphicx}
\usepackage{epstopdf}
\usepackage{pdfpages}
\usepackage{caption}
\usepackage[list=true]{subcaption}
\usepackage{float}
\usepackage{standalone}
\usepackage{import}
\usepackage{tocloft}
\usepackage{wrapfig}
\usepackage{authblk}
\usepackage{array}
\usepackage{booktabs}
\usepackage[toc,page,title,titletoc]{appendix}
\usepackage{xunicode}
\usepackage{fontspec}
\usepackage{pgfplots}
\usepackage{SIunits}
\pgfplotsset{compat=newest}
\pgfplotsset{plot coordinates/math parser=false}
\newlength\figureheight 
\newlength\figurewidth
\usepackage{unicode-math}
\usepackage[
    backend=bibtexu,
	texencoding=utf8,
bibencoding=utf8,
    style=ieee,
    sortlocale=nl_NL,
    language=auto
]{biblatex}
\usepackage{listings}
\newcommand{\includecode}[3][c]{\lstinputlisting[caption=#2, escapechar=, style=#1]{#3}}
\newcommand{\superscript}[1]{\ensuremath{^{\textrm{#1}}}}
\newcommand{\subscript}[1]{\ensuremath{_{\textrm{#1}}}}


\newcommand{\chapternumber}{\thechapter}
\renewcommand{\appendixname}{Bijlage}
\renewcommand{\appendixtocname}{Bijlagen}
\renewcommand{\appendixpagename}{Bijlagen}

\usepackage[hidelinks]{hyperref} %<--------ALTIJD ALS LAATSTE


\begin{document}


\begin{titlepage}

\newcommand{\HRule}{\rule{\linewidth}{0.5mm}} % Defines a new command for the horizontal lines, change thickness here

\center % Center everything on the page
 
%----------------------------------------------------------------------------------------
%	HEADING SECTIONS
%----------------------------------------------------------------------------------------

\textsc{\LARGE TU Delft}\\[1.5cm] % Name of your university/college
\textsc{\Large EE28 EPO-3 Moduleopdracht 5}\\[0.5cm] % Major heading such as course name
\textsc{\large Klokbuffer}\\[0.5cm] % Minor heading such as course title

%----------------------------------------------------------------------------------------
%	TITLE SECTION
%----------------------------------------------------------------------------------------

\HRule \\[0.4cm]
{ \huge \bfseries Klokbuffer  noodzakelijk voor kloknetwerk?}\\[0.4cm] % Title of your document
\HRule \\[1.5cm]
 
%----------------------------------------------------------------------------------------
%	AUTHOR SECTION
%----------------------------------------------------------------------------------------

\begin{minipage}{0.4\textwidth}
\begin{flushleft} \large
\emph{Author:}\\
Kees \textsc{Hogenhout} 4220927\\ % Your name 
Tu \textsc{Hoang} 4203496 %other name
\end{flushleft}
\end{minipage}
~
\begin{minipage}{0.4\textwidth}
\begin{flushright} \large
\emph{Supervisor:} \\
Dr. Marjam \textsc{Bartek} % Supervisor's Name
\end{flushright}
\end{minipage}\\[4cm]

% If you don't want a supervisor, uncomment the two lines below and remove the section above
%\Large \emph{Author:}\\
%John \textsc{Smith}\\[3cm] % Your name

%----------------------------------------------------------------------------------------
%	DATE SECTION
%----------------------------------------------------------------------------------------

{\large \today}\\[3cm] % Date, change the \today to a set date if you want to be precise

%----------------------------------------------------------------------------------------
%	LOGO SECTION
%----------------------------------------------------------------------------------------

%\includegraphics{Logo}\\[1cm] % Include a department/university logo - this will require the graphicx package
 
%----------------------------------------------------------------------------------------
% 

\textbf{Abstract} \\
In dit onderzoekje wordt uitgezocht in welke mate een klokbuffer nodig is voor een netwerk van 125 D-flipflops. Eerst wordt de t-90\% van een inverter bepaald met een C\tss{load} van 25pF (belastingscapaciteit van die 125 DFF's). Vervolgens worden er drie inverters (die groter zijn)  in cascade geplaatst met dezelfde belastingscapaciteit en wordt weer de t-90\% bepaald. Te zien is aan de simulatieresultaten dat de drie inverters als buffer werken. De afvlakking bij een 0-1 overgang van de uitgang is namelijk heel klein, terwijl die afvlakking bij de meting met een enkele transistor veel groter is. 

\vfill % Fill the rest of the page with whitespace

\end{titlepage}
\end{document}

\chapter{Inleiding}
In de inleiding beschrijf je het probleem wat je op wilt lossen, zonder al te diep hierop in te gaan. Je
geeft hier ook de motivatie en context van je werk (waar is het belangrijk voor, welk groter probleem
wordt hiermee opgelost of geadresseerd).

\chapter{Probleemstelling}
De probleemstelling kan bij een beknopt rapport als deze wellicht volledig in de inleiding verwerkt
worden. In sommige gevallen is het beter om een letterlijke kopie van de opdracht op te nemen, in
andere gevallen is het beter om de opdracht in eigen woorden op te nemen met daaromheen je eigen
interpretatie. Wanneer de opdracht niet voldoende eenduidig gespecificeerd is, bijvoorbeeld, moet je
aangeven hoe je die verder ingeperkt hebt.

\chapter{Aanpak}
De kern van het rapport kan uiteen vallen in verschillende delen, bijvoorbeeld een eerste gedeelte
met achtergrondkennis en -theorie, en een volgend gedeelte met je eigen werk. Het eerste stuk
heeft dan tot doel om de theorie zodanig samen te vatten dat de rest van het rapport gemakkelijker
begrepen kan worden. Het hoofdgedeelte kan uit meerdere secties bestaan, het hangt heel sterk van
het onderwerp af wat dan de beste indeling is. Een min of meer gebruikelijke indeling is achtergrond
kennis - methode (opzet van experimenten) - resultaten, maar dat is niet meer dan een vuistregel.



\section{Theorie}
\section{Eisen/randvoorwaarden voor de simulatie}
\section{Concrete aanpak}

\chapter{Resultaten}

\section{Plain results}



\begin{table} [h!]
\centering % 
\begin{tabular}{l c c c c c} 
\toprule % Top horizontal line
& \multicolumn {4}{c}{Formula variables} \\ % Amalgamating several columns into one cell is done using the \multicolumn command as seen on this line
\cmidrule(l){1-3} % Horizontal line spanning less than the full width of the table - you can add (r) or (l) just before the opening curly bracket to shorten the rule on the left or right side
L($\mu$m)&  W($\mu$m)  & C\tss{g}\\ % Column names row
\midrule % In-table horizontal line
2.2 & 3.0  & 8.3381 \\ % Content row 1
2.4 & 3.0 & 9.1672 \\ % Content row 2
2.8 & 3.0 & 10.825 \\ % Content row 3
3.2 & 3.0 & 12.483 \\ % Content row 4
3.6 & 3.0 & 14.141 \\ % Content row 5
4.0 & 3.0 & 15.799 \\ % Content row 6
4.4 & 3.0 & 17.457 \\ % Content row 7
4.8 & 3.0 & 19.115 \\ % Content row 8
\midrule % In-table horizontal line
\midrule % In-table horizontal line
%Average Rate & 0.920 & 0.882 \\ % Summary/total row
\bottomrule % Bottom horizontal line
\end{tabular}
\caption{Table caption text} % Table caption, can be commented out if no caption is required
\label{tab:template} % A label for referencing this table elsewhere, references are used in text as \ref{label}
\end{table}




\section{Analyse}
\chapter {Discussie}
In de ‘discussie’ sectie reflecteer je op de resultaten. Een uitkomst zonder discussie en/of reflectie
heeft zelden ‘impact’. Welke doelen zijn wel bereikt, welke niet. Komt je resultaat overeen met de
verwachte uitkomst? Zo niet, wat zou de oorzaak van het verschil kunnen zijn? Wat zijn de sterkepunten van je werk, en welke de zwakke? Wat zou je een andere keer anders doen? Bespreek, indien
van toepassing, implicaties, van je werk. Voorbeeld: wat is het mogelijke gevolg van Neutrino’s die
sneller reizen dan het licht? Wat zijn mogelijke bronnen van meetfouten of interpretatiefouten?

\chapter{Conclusie}
De conclusie is zeer belangrijk. Deze is meestal relatief kort, en vat heel beknopt de resultaten en
belangrijkste reflectie daarop nog eens samen. Wanneer de conclusie te lang zou worden, is het beter
om een aparte discussie sectie (zie hierboven) toe te voegen. In de conclusie kun je, indien van
toepassing, aanbevelingen doen voor verder werk/onderzoek. (Indien dat te lang zou worden, kun je
daar ook weer een aparte sectie van maken.) In de conclusie maak je ook eventuele slotopmerkingen.
Hou er rekening mee dat er lezers zijn die alleen het abstract en de conclusie lezen, en maak deze
stukken ook voor hen nuttig, bijvoorbeeld om te beslissen of ze het gehele rapport gaan lezen

\section{Conclusie}
\section{Discussie}



\chapter{Appendix}
\section{Matlab code}
\subsection{Matlab code om Cg vs. L te plotten}
\label{M1}
\lijst{inputfiles/c0.m}

\subsection{Bereken de dikte van het gate-oxide}
\label{M2}
\lijst{inputfiles/gateoxide.m}

\end{document}