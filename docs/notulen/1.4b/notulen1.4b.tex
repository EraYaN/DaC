


\documentclass{article}
% Include all project wide packages here.
\usepackage{fullpage}
\usepackage{polyglossia}
\setmainlanguage{dutch}
\usepackage{csquotes}
\usepackage{graphicx}
\usepackage{epstopdf}
\usepackage{pdfpages}
\usepackage{caption}
\usepackage[list=true]{subcaption}
\usepackage{float}
%\usepackage{mathtools}
\usepackage{standalone}
\usepackage{import}
\usepackage{tocloft}
\usepackage{wrapfig}
\usepackage{authblk}
\usepackage{array}
\usepackage{booktabs}
\usepackage[toc,page,title,titletoc]{appendix}
\usepackage{xunicode}
\usepackage{amsmath}
\usepackage{fontspec}
\usepackage{unicode-math}
\usepackage[
    backend=bibtexu,
	texencoding=utf8,
bibencoding=utf8,
    style=ieee,
    sortlocale=nl_NL,
    language=auto
]{biblatex}
\usepackage{listings}
\newcommand{\includecode}[3][c]{\lstinputlisting[caption=#2, escapechar=, style=#1]{#3}}
\newcommand{\superscript}[1]{\ensuremath{^{\textrm{#1}}}}
\newcommand{\subscript}[1]{\ensuremath{_{\textrm{#1}}}}


\newcommand{\chapternumber}{\thechapter}
\renewcommand{\appendixname}{Bijlage}
\renewcommand{\appendixtocname}{Bijlagen}
\renewcommand{\appendixpagename}{Bijlagen}

\usepackage[hidelinks]{hyperref} %<--------ALTIJD ALS LAATSTE

\renewcommand{\familydefault}{\sfdefault}

\setmainfont[Ligatures=TeX]{Myriad Pro}
\setmathfont{Asana Math}
\setmonofont{Lucida Console}

\usepackage{titlesec, blindtext, color}
\definecolor{gray75}{gray}{0.75}
\newcommand{\hsp}{\hspace{20pt}}
\titleformat{\chapter}[hang]{\Huge\bfseries}{\chapternumber\hsp\textcolor{gray75}{|}\hsp}{0pt}{\Huge\bfseries}
\renewcommand{\familydefault}{\sfdefault}
\renewcommand{\arraystretch}{1.2}
\setlength\parindent{0pt}

%For code listings
\definecolor{black}{rgb}{0,0,0}
\definecolor{browntags}{rgb}{0.65,0.1,0.1}
\definecolor{bluestrings}{rgb}{0,0,1}
\definecolor{graycomments}{rgb}{0.4,0.4,0.4}
\definecolor{redkeywords}{rgb}{1,0,0}
\definecolor{bluekeywords}{rgb}{0.13,0.13,0.8}
\definecolor{greencomments}{rgb}{0,0.5,0}
\definecolor{redstrings}{rgb}{0.9,0,0}
\definecolor{purpleidentifiers}{rgb}{0.01,0,0.01}


\lstdefinestyle{csharp}{
language=[Sharp]C,
showspaces=false,
showtabs=false,
breaklines=true,
showstringspaces=false,
breakatwhitespace=true,
escapeinside={(*@}{@*)},
columns=fullflexible,
commentstyle=\color{greencomments},
keywordstyle=\color{bluekeywords}\bfseries,
stringstyle=\color{redstrings},
identifierstyle=\color{purpleidentifiers},
basicstyle=\ttfamily\small}

\lstdefinestyle{c}{
language=C,
showspaces=false,
showtabs=false,
breaklines=true,
showstringspaces=false,
breakatwhitespace=true,
escapeinside={(*@}{@*)},
columns=fullflexible,
commentstyle=\color{greencomments},
keywordstyle=\color{bluekeywords}\bfseries,
stringstyle=\color{bluestrings},
identifierstyle=\color{purpleidentifiers}
}

\lstdefinestyle{vhdl}{
language=VHDL,
showspaces=false,
showtabs=false,
breaklines=true,
showstringspaces=false,
breakatwhitespace=true,
escapeinside={(*@}{@*)},
columns=fullflexible,
commentstyle=\color{greencomments},
keywordstyle=\color{bluekeywords}\bfseries,
stringstyle=\color{redstrings},
identifierstyle=\color{purpleidentifiers}
}

\lstdefinestyle{xaml}{
language=XML,
showspaces=false,
showtabs=false,
breaklines=true,
showstringspaces=false,
breakatwhitespace=true,
escapeinside={(*@}{@*)},
columns=fullflexible,
commentstyle=\color{greencomments},
keywordstyle=\color{redkeywords},
stringstyle=\color{bluestrings},
tagstyle=\color{browntags},
morestring=[b]",
  morecomment=[s]{<?}{?>},
  morekeywords={xmlns,version,typex:AsyncRecords,x:Arguments,x:Boolean,x:Byte,x:Char,x:Class,x:ClassAttributes,x:ClassModifier,x:Code,x:ConnectionId,x:Decimal,x:Double,x:FactoryMethod,x:FieldModifier,x:Int16,x:Int32,x:Int64,x:Key,x:Members,x:Name,x:Object,x:Property,x:Shared,x:Single,x:String,x:Subclass,x:SynchronousMode,x:TimeSpan,x:TypeArguments,x:Uid,x:Uri,x:XData,Grid.Column,Grid.ColumnSpan,Click,ClipToBounds,Content,DropDownOpened,FontSize,Foreground,Header,Height,HorizontalAlignment,HorizontalContentAlignment,IsCancel,IsDefault,IsEnabled,IsSelected,Margin,MinHeight,MinWidth,Padding,SnapsToDevicePixels,Target,TextWrapping,Title,VerticalAlignment,VerticalContentAlignment,Width,WindowStartupLocation,Binding,Mode,OneWay,xmlns:x}
}

%defaults
\lstset{
basicstyle=\ttfamily\small,
extendedchars=false,
numbers=left,
numberstyle=\ttfamily\tiny,
stepnumber=1,
tabsize=4,
numbersep=5pt
}

\begin{document}

\title{Notulen EPO-2 D-2 week }%<---WEEK NR
\author{}%<----AUTEUR
\maketitle

\section*{Informatie}
Datum van vergadering: \today \\ %<----DATUM
Locatie van vergadering: TU Delft - Drebbelweg zaal 0.210 %<----LOCATIE
\subsection*{Genodigden}
\begin{center}
\begin{tabular}{|c |c | c| c|}
	%%VUL AANWEZIGHEID IN: "Aanwezig", "Afwezig", "Te laat"
	\hline
Efraïm Eland & Aanwezig & Erwin de Haan & Aanwezig \\
	\hline
Robin Hes (voorzitter) & Aanwezig & Tu Hoang & Aanwezig \\
	\hline
Kees Hogenhout (notulist) & Aanwezig & Alex Janssen & Aanwezig\\
	\hline
Jorden Kerkhof & Aanwezig (pas 10:30) & Peter Stijnman & Aanwezig \\
	\hline
Xenia Wesdijk & Aanwezig & & \\
	\hline
\end{tabular}
\end{center}

\section*{Vergadering}
\begin{enumerate}
	
	%%%%%%%%%%%%%%%%%%%%%%%
	%----- VOORAF
	\subsection*{Vooraf}
	\item Vergadering geopend om 9:05  %<----OPENINGSTIJD
	\item Notulen vorige vergadering
	\begin{itemize}
		%%NOTULEN VORIGE VERGADERING
		\item Deze bleken nog niet in Git te staan, wel in dropbox. Inmiddels staan ze er wel in. Notulen akkoord.
	\end{itemize}
	\item Moduleontwerp + verslag - woensdag 2 oktober
	\item Mededelingen
	\begin{itemize}
		%%MEDEDELINGEN
		\item -
	\end{itemize}

	%%%%%%%%%%%%%%%%%%%%%%%
	%----- DEADLINES
	\subsection*{Deadlines}
	\item Verstreken deadlines
	\begin{itemize}
		%%VERSTREKEN DEADLINES
		\item Vorige week vrijdag: inleveren FinFet opdracht.
	\end{itemize}
	\item Komende deadlines
	\begin{itemize}
		%%KOMENDE DEADLINES
		\item Moduleontwerp + verslag - woensdag 2 oktober
	\end{itemize}

	%%%%%%%%%%%%%%%%%%%%%%%
	%----- AGENDAPUNTEN
	\subsection*{Agendapunten}
	De agenda omvatte slechts de deadline en de To Do lijst voor vandaag. 

	%%%%%%%%%%%%%%%%%%%%%%%
	%----- ACTIES
	\subsection*{Actiepunten}
	\item Todo vandaag
	\begin{itemize}
		%%TODO VANDAAG
		\item Kees en Jorden (voorzover hij nog komt): kloppende layout maken en verslag
		\item Robin, Xenia en Peter: verslag Moduleontwerp
		\item Erwin en Tu: afmaken RAM en verslag (indien mogelijk)
	\end{itemize}
	\item Todo later
	\begin{itemize}
		%%TODO LATER
		\item Alles is in GoWithTheFlow, dus er kan niet veel thuis gedaan worden, op het verslag na. 
	\end{itemize}

	%%%%%%%%%%%%%%%%%%%%%%%
	%----- BESLUITEN
	\subsection*{Besluiten}
	\item Genomen besluiten
	\begin{itemize}
		%%GENOMEN BESLUITEN
		\item-
	\end{itemize}

	%%%%%%%%%%%%%%%%%%%%%%%
	%----- AFSLUITING
	\noindent 
	\subsection*{Afsluitend}
	\item Vergadering gesloten om 12:39%<----SLUITINGSTIJD
	\item Nieuwe inbreng van Efraïm: probeer wat toe te voegen aan je verslag, boven de normale specificaties. Dit levert meer diepgang op (en een hoger cijfer). 
	\item Iedereen, behalve Erwin en Tu, is klaar met de layout en bezig met het verslag. 
	\item Voor de volgende keer: lees alvast opdracht 4 door en het stuk in Geïntegreerde Schakelingen over de 4 werkgebieden van de transistor.
	\item Volgende keer is Robin weer voorzitter en Jorden is notulist. 

\end{enumerate}

\end{document}