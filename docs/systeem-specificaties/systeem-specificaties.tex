\documentclass{scrartcl} 
% Include all project wide packages here.
\usepackage{fullpage}
\usepackage{polyglossia}
\setmainlanguage{dutch}
\usepackage{csquotes}
\usepackage{graphicx}
\usepackage{epstopdf}
\usepackage{pdfpages}
\usepackage{caption}
\usepackage[list=true]{subcaption}
\usepackage{float}
%\usepackage{mathtools}
\usepackage{standalone}
\usepackage{import}
\usepackage{tocloft}
\usepackage{wrapfig}
\usepackage{authblk}
\usepackage{array}
\usepackage{booktabs}
\usepackage[toc,page,title,titletoc]{appendix}
\usepackage{xunicode}
\usepackage{amsmath}
\usepackage{fontspec}
\usepackage{unicode-math}
\usepackage[
    backend=bibtexu,
	texencoding=utf8,
bibencoding=utf8,
    style=ieee,
    sortlocale=nl_NL,
    language=auto
]{biblatex}
\usepackage{listings}
\newcommand{\includecode}[3][c]{\lstinputlisting[caption=#2, escapechar=, style=#1]{#3}}
\newcommand{\superscript}[1]{\ensuremath{^{\textrm{#1}}}}
\newcommand{\subscript}[1]{\ensuremath{_{\textrm{#1}}}}


\newcommand{\chapternumber}{\thechapter}
\renewcommand{\appendixname}{Bijlage}
\renewcommand{\appendixtocname}{Bijlagen}
\renewcommand{\appendixpagename}{Bijlagen}

\usepackage[hidelinks]{hyperref} %<--------ALTIJD ALS LAATSTE

\renewcommand{\familydefault}{\sfdefault}

\setmainfont[Ligatures=TeX]{Myriad Pro}
\setmathfont{Asana Math}
\setmonofont{Lucida Console}

\usepackage{titlesec, blindtext, color}
\definecolor{gray75}{gray}{0.75}
\newcommand{\hsp}{\hspace{20pt}}
\titleformat{\chapter}[hang]{\Huge\bfseries}{\chapternumber\hsp\textcolor{gray75}{|}\hsp}{0pt}{\Huge\bfseries}
\renewcommand{\familydefault}{\sfdefault}
\renewcommand{\arraystretch}{1.2}
\setlength\parindent{0pt}

%For code listings
\definecolor{black}{rgb}{0,0,0}
\definecolor{browntags}{rgb}{0.65,0.1,0.1}
\definecolor{bluestrings}{rgb}{0,0,1}
\definecolor{graycomments}{rgb}{0.4,0.4,0.4}
\definecolor{redkeywords}{rgb}{1,0,0}
\definecolor{bluekeywords}{rgb}{0.13,0.13,0.8}
\definecolor{greencomments}{rgb}{0,0.5,0}
\definecolor{redstrings}{rgb}{0.9,0,0}
\definecolor{purpleidentifiers}{rgb}{0.01,0,0.01}


\lstdefinestyle{csharp}{
language=[Sharp]C,
showspaces=false,
showtabs=false,
breaklines=true,
showstringspaces=false,
breakatwhitespace=true,
escapeinside={(*@}{@*)},
columns=fullflexible,
commentstyle=\color{greencomments},
keywordstyle=\color{bluekeywords}\bfseries,
stringstyle=\color{redstrings},
identifierstyle=\color{purpleidentifiers},
basicstyle=\ttfamily\small}

\lstdefinestyle{c}{
language=C,
showspaces=false,
showtabs=false,
breaklines=true,
showstringspaces=false,
breakatwhitespace=true,
escapeinside={(*@}{@*)},
columns=fullflexible,
commentstyle=\color{greencomments},
keywordstyle=\color{bluekeywords}\bfseries,
stringstyle=\color{bluestrings},
identifierstyle=\color{purpleidentifiers}
}

\lstdefinestyle{vhdl}{
language=VHDL,
showspaces=false,
showtabs=false,
breaklines=true,
showstringspaces=false,
breakatwhitespace=true,
escapeinside={(*@}{@*)},
columns=fullflexible,
commentstyle=\color{greencomments},
keywordstyle=\color{bluekeywords}\bfseries,
stringstyle=\color{redstrings},
identifierstyle=\color{purpleidentifiers}
}

\lstdefinestyle{xaml}{
language=XML,
showspaces=false,
showtabs=false,
breaklines=true,
showstringspaces=false,
breakatwhitespace=true,
escapeinside={(*@}{@*)},
columns=fullflexible,
commentstyle=\color{greencomments},
keywordstyle=\color{redkeywords},
stringstyle=\color{bluestrings},
tagstyle=\color{browntags},
morestring=[b]",
  morecomment=[s]{<?}{?>},
  morekeywords={xmlns,version,typex:AsyncRecords,x:Arguments,x:Boolean,x:Byte,x:Char,x:Class,x:ClassAttributes,x:ClassModifier,x:Code,x:ConnectionId,x:Decimal,x:Double,x:FactoryMethod,x:FieldModifier,x:Int16,x:Int32,x:Int64,x:Key,x:Members,x:Name,x:Object,x:Property,x:Shared,x:Single,x:String,x:Subclass,x:SynchronousMode,x:TimeSpan,x:TypeArguments,x:Uid,x:Uri,x:XData,Grid.Column,Grid.ColumnSpan,Click,ClipToBounds,Content,DropDownOpened,FontSize,Foreground,Header,Height,HorizontalAlignment,HorizontalContentAlignment,IsCancel,IsDefault,IsEnabled,IsSelected,Margin,MinHeight,MinWidth,Padding,SnapsToDevicePixels,Target,TextWrapping,Title,VerticalAlignment,VerticalContentAlignment,Width,WindowStartupLocation,Binding,Mode,OneWay,xmlns:x}
}

%defaults
\lstset{
basicstyle=\ttfamily\small,
extendedchars=false,
numbers=left,
numberstyle=\ttfamily\tiny,
stepnumber=1,
tabsize=4,
numbersep=5pt
}
\addbibresource{../library/bibliography.bib}

\author{Groep 1 (DaC)}
\title{EPO-3 "Maak een Chip" Systeem Specificaties}

\date{17-10-2013}

\begin{document}
\pagenumbering{roman}
\maketitle
\vspace{80 mm}
\section*{Samenvatting}
In dit verslag zal de opzet van het project uitgelegd worden. Het uiteindelijke resultaat van dit project is het maken van een functionele chip, waarbij de chip volledig zal worden gespecificeerd, ontworpen en gesimuleerd door en groep studenten. De chip zal gefabriceerd worden door DIMES. 
\newpage
\setlength{\cftbeforetoctitleskip}{-3em}

\tableofcontents
\newpage
\pagenumbering{arabic}
\section{Specificaties}
\subsection {Functionele Eisen}
\begin {itemize}
\item Minimale beeldresolutie 160x120  in 16  kleuren.
\item Minimaal 2 vertexes te verbinden met een lijn.
\item Interfacing met een AVR zodat een PC systeem gesimuleerd wordt. Met de AVR als CPU en RAM en onze chip als GPU en VRAM.
\item VGA signaal uitsturen met extrene of interne DAC.
\end{itemize}


\subsection {specificaties van de IC}
\begin {itemize}
\item Max 2 bond bars (40000 transistorparen) per bond bar 32 aansluitingen.
\item Voor de FSM’s (Finite State Machine) mogen alleen die van het Moore-type gebruikt worden.
\item Als de schakeling geactiveerd wordt moeten alle FSM’s in hun begintoestand komen door middel van een reset signaal.
\item Voor de opwekking van het kloksignaal kan gebruik gemaakt worden van een kristal van 6.144 MHz.
\item Het streven is om zo weinig mogelijk componenten extern te gebruiken. De dissipatie van de
chip dient echter ook beperkt te zijn. Dit geeft een compromis voor de maximale stroom die de elektronica mag dissiperen voor de aansturing van de LEDs, etc.
\item De voedingsspanning van het IC bedraagt 5 Volt.

\end{itemize}

\subsection {specificaties van de arduino}
\begin {itemize}
\item Voedingsspanning:	 5V
\item Input spanning: 	7-12V
\item Digital I/O Pins: 	54 (of which 15 provide PWM output)
\item Analog Input Pins:	16
\item Flash Memory	:	256 KB 
\item SRAM:			8 KB
\item EEPROM:		4 KB
\item Clock Speed:		16 MHz
\end {itemize}

\subsection { SPI controller}

Voor de communicatie tussen de arduïno en onze gpu hebben we een SPI decoder nodig. De ingang SS staat voor slave select, dit wordt normaal gebruikt voor het geval dat je meerdere slave chips hebt, aangezien dit bij ons niet het geval is zal dit signaal altijd hoog zijn. Het signaal SCLK is voor het CLK signaal voor de communicatie. Als laatste ingang is er de MOSI, dit staat voor master out en slave in, dit is de daadwerkelijke data die verzonden wordt van de arduïno naar onze gpu. Voor de uitgangen hebben we MISO, master in slave out, dit is om de volgende instructie aan te vragen van de arduïno. Dan zijn er nog de uitgangen naar de instructie decoder, de 8 bits data bus voor om alle data te sturen als de data ready hoog is.

\begin {itemize}
\item ingangen:
\begin {itemize}
\item MOSI 
\item SS (1 bit altijd hoog)
\item SCLK (clk signaal)
\item readytosend (1 bit)
\end {itemize}
\item uitgangen:
\begin {itemize}
\item MISO
\item 8 bits data bus
\item data ready
\end {itemize}
\end {itemize}

\subsection { instructie decoder}

De instructiedecoder zorgt ervoor dat de data die vanuit de arduïno komt vertaald wordt voor de rest van de gpu. 
Dit houdt in dat de intructiedecoder het 'drawblok' van die juiste informatie voorziet zodat deze de pixels kan tekenen en dat het in het stateregister opslaat in welke screenbuffer er geschreven wordt.
De instructiedecoder heeft een 8bits bus ingang van de SPI decoder, deze wordt ingelezen als de ingang data ready hoog is. Als de 8 bits zijn ingelezen stuurt de instructiedecoder een signaal terug dat aangeeft dat de volgende 8 bits verstuurd mogen worden.
Wanneer dan de complete instructie, 3 keer 8 bits, verstuurd is zal er een signaal worden gegeven dat de volgende instructie gestuurd kan worden.
Als uitgangen van de instructiedecoder zijn er de coördinaten van het te tekenen object plus de kleur die dit object moet hebben, deze signalen gaan natuurlijk naar het 'drawblok'.
Om de juiste module van het 'drawblok' aan te sturen wordt gebruik gemaakt van een enabled signaal dat aangeeft welke module de instructie moet uitvoeren.
Van het 'drawblok' krijgt de instructiedecoder een signaal als het klaar is met tekenen zodat deze weer een volgend object kan tekenen. 
Verder wordt de instructiedecoder gebruikt om te bepalen welke screenbuffer er gebruikt wordt om op te schrijven en dus indirect welke om te lezen. Deze informatie wordt verstuurd naar een stateregister doormiddel van een signaal set en de data voor de juiste screenbuffer.
En dit alles wordt gedaan synchroon met de klok.

\begin {itemize}
\item ingangen:
\begin{itemize}
\item data clock  
\item 8 bits bus 
\item data ready (1bit)
\item done (1 bit)
\end{itemize}

\item uitgangen:
\begin{itemize}
\item readytosend (1 bit)
\item instuctieready (1 bit).
\item kleur (4 bits)
\item x coördinaat (8 bits)
\item y coördinaat (7 bits)
\item enabled (1 bit)
\item set (1 bit)
\item data (1 bit)
\end{itemize}
\end{itemize}


\subsection { draw logic}

In dit blok worden de instructie omgezet naar pixels, vierkantjes, lijnen en cirkels. Dit blok krijgt de informatie binnen van de instructie decoder, deze geeft aan welke module er gebruikt moet worden met de bijbehorende coördinaten en kleur.
De keuze voor welke module er gebruikt moet worden gebeurd doormiddel van het signaal enabled. verder neemt het draw blok het signaal activescreenbuffer mee zodat er in de juiste screenbuffer geschreven wordt. Tevens krijgt het 'drawblok' het signaal RAMclaimed binnen van de VGA controller, dit houdt in dat als de VGA controller aan het lezen is in het RAM dat het 'drawblok' er niet in kan schrijven. Als laatste ingang krijgt het 'drawblok' een signaal write binnen van de RAM controller dat zegt wanneer er geschreven mag worden.
Als uitgangen heeft het 'drawblok' het signaal done als de module klaar is met het tekenen van een de instructie zodat er een volgende instructie kan doorgegeven worden. Verder heeft het de uitgangen naar de RAM controller, hierin worden de kleur 

\begin {itemize}
\item ingangen:
\begin {itemize}
\item kleur (4 bits)
\item x coördinaat (8 bits)
\item y coördinaat (7 bits)
\item enabled (1 bit)
\item activescreenbuffer (1 bit)
\item RAMclaimed (1 bit)
\item write (1 bit)
\end {itemize}

\item uitgangen:
\begin {itemize}
\item kleur (4 bits)
\item adres (16 bits)
\item done (1 bit)
\end {itemize}


\subsubsection { pixel}

Hierin worden aparte pixels getekend, hiervoor zijn de x,y coördinaten en de kleur nodig.


\subsubsection { square}

In deze module wordt een vierkant getekend met de x,y coördinaten van de eerste pixel links boven en de laatste pixel rechtsonder van het vierkant en natuurlijk de kleur die het vierkant moet hebben.

\subsubsection { lijn}

Hier wordt het algoritme van Bresenham gebruikt hiervoor hebben we de x,y coördinaten nodig van het begin en het eindpunt van de lijn.

\subsubsection { cirkel}

In deze laatste module gebruiken we het midpoint circle algoritme met x1 als de radius.


\subsection { ram controller}

Omdat we een externe RAM gebruiken voor onze screenbuffers is het nodig om hiervoor communicatie op te stellen, dit zodat er in het juiste gedeelte van de RAM geschreven en gelezen wordt en dat dit niet tegelijkertijd gebeurd omdat het RAM dat niet aankan. We hebben voor een externe RAM gekozen omdat we nooit een RAM geheugen groot genoeg konden maken met de benodigde transistoren. De RAM controller stuurt naar het 'drawblok'een signaal write als het mag gaan schrijven in het RAM geheugen, de RAM controller krijgt dan de kleur en het adress binnen van het 'drawblok'. Als de VGA controller mag gaan lezen krijgt het een signaal read binnen van de RAM controller dan zal er weer data naar de VGA controller verstuurd worden, kleur en adres.

\begin {itemize}
\item ingangen:
\begin {itemize}
\item kleur (4 bits)
\item adres (16 bits)
\end {itemize}
\item uitgangen:
\begin {itemize}
\item write (1 bit)
\item read (1 bit)
\item kleur (4 bits)
\item adres (16 bits)
\end {itemize}
\end {itemize}


\subsection { vga controller}

De VGA controller is er om de screenbuffer uit te lezen en de data die daarin staat naar het beeldscherm te sturen. Het leest van de screenbuffer waarop dat moment niet in geschreven wordt, de inverse van active screen buffer dus. als de controller het RAM geheugen claimt met het signaal RAMclaimed mag er niet worden geschreven in het RAM. De VGA controller leest dan de signalen h,v sync (de positie van de pixel) en de waardes van RGBI ( rood , groen, blauw en de intensiteit). Het uitlezen van de screenbuffer kost 55 ns en is dus snel genoeg aangezien een klokpuls 160 ns is. In de overige tijd kan dus worden geschreven in het RAM.

\begin {itemize}
\item ingangen:
\begin {itemize}
\item active screen buffer (1 bit)
\item read (1 bit)
\item kleur (4 bits)
\item adres (16 bits)
\end {itemize}
\item uitgangen:
\begin {itemize}
\item RAMclaimed (1 bit)
\item hsync (1 pin)
\item vsync (1 pin)
\item RGBI (4 pinnen)
\end {itemize}
\end {itemize}

\end{document}
