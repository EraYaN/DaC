%=========================================================================
\documentclass[11pt,twoside,a4paper]{article}
\usepackage[dutch]{babel}
\usepackage{a4wide,times}
\title{Titel van het Ontwerp}
\author{
Gemaakt door\\
Naam Student1, 3458923\\
Naam Student2, 3456738\\
\\
In opdracht van \\
Naam Student3, 3458966\\
Naam Student4, 3456706\\
}
\date{21 September 2013}
\begin{document}
\maketitle
\thispagestyle{empty}
\vspace{30 mm}
\begin{center}
\Large \bf 
Samenvatting
\end{center}
Dit document beschrijft een indeling voor een ontwerprapport over de moduleopdracht bij het project EE2821 en kan worden gebruikt als een sjabloon bij het schrijven van een zo'n ontwerprapport.
Hier dien je dan een samenvatting te geven van wat in het rapport beschreven wordt.
\clearpage

\tableofcontents
\clearpage

\section{Inleiding}
Vertel iets over de context van deze opdracht.

\section{Specificatie}
Geef hierbij o.a. de letterlijke beschrijving van de opdrachtgevers.

\section{Ontwerp}
Beschrijf hoe de beschrijving van de opdrachtgevers verder geïnterpreteerd is en hoe het ontwerp gedaan is, tot aan de VHDL beschrijving.
\begin{itemize}
\item
Blokrepresentatie met in en uitgangen
\item
Eventuele opdeling in deelschakelingen
\item
Een of meerdere toestandsdiagrammen 
\item
VHDL behaviour beschrijving van de schakeling.
\item
Simulatieresultaten van de VHDL behaviour beschrijving.
\end{itemize}

\section{Implementatie}
Vertel over de implementatie van de schakeling:
\begin{itemize}
\item
Hoe ging de synthese ?
\item
Layoutresultaat.
\item
Switch-level/Spice simulatieresultaten van de uit de layout ge-extraheerde schakeling.
\end{itemize}

\section{Conclusies}
Voldoet het ontwerp aan de specificatie ?
Wat zou er eventueel verbeterd kunnen worden ?

\begin{thebibliography}{9}
\bibitem{labelboek1}
Auteur1, 
Boek1, 
Uitgeverij, 
jaar.
\bibitem{labelTitel}
Auteur1, 
Titel, 
Tijdschrift, 
Vol. 1, 
Nr. 1, 
jaar, 
pp. 12-15.
\bibitem{labelWeb}
Webpagina titel, 
http link, 
geraadpleegd op 19 sep. 2013.
\end{thebibliography}
\end{document}
