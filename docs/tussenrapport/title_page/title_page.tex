\documentclass[12pt]{scrreprt}
\newcommand{\tss}{\textsubscript}
% Include all project wide packages here.
\usepackage{fullpage}
\usepackage{polyglossia}
\setmainlanguage{dutch}
\usepackage{csquotes}
\usepackage{graphicx}
\usepackage{epstopdf}
\usepackage{pdfpages}
\usepackage{caption}
\usepackage[list=true]{subcaption}
\usepackage{float}
\usepackage{standalone}
\usepackage{import}
\usepackage{tocloft}
\usepackage{wrapfig}
\usepackage{authblk}
\usepackage{array}
\usepackage{booktabs}
\usepackage[toc,page,title,titletoc]{appendix}
\usepackage{xunicode}
\usepackage{fontspec}
\usepackage{pgfplots}
\usepackage{SIunits}
\pgfplotsset{compat=newest}
\pgfplotsset{plot coordinates/math parser=false}
\newlength\figureheight 
\newlength\figurewidth
\usepackage{unicode-math}
\usepackage[
    backend=bibtexu,
	texencoding=utf8,
bibencoding=utf8,
    style=ieee,
    sortlocale=nl_NL,
    language=auto
]{biblatex}
\usepackage{listings}
\newcommand{\includecode}[3][c]{\lstinputlisting[caption=#2, escapechar=, style=#1]{#3}}
\newcommand{\superscript}[1]{\ensuremath{^{\textrm{#1}}}}
\newcommand{\subscript}[1]{\ensuremath{_{\textrm{#1}}}}


\newcommand{\chapternumber}{\thechapter}
\renewcommand{\appendixname}{Bijlage}
\renewcommand{\appendixtocname}{Bijlagen}
\renewcommand{\appendixpagename}{Bijlagen}

\usepackage[hidelinks]{hyperref} %<--------ALTIJD ALS LAATSTE


\begin{document}


\begin{titlepage}

\newcommand{\HRule}{\rule{\linewidth}{0.5mm}} % Defines a new command for the horizontal lines, change thickness here

\center % Center everything on the page
 
%----------------------------------------------------------------------------------------
%	HEADING SECTIONS
%----------------------------------------------------------------------------------------

\textsc{\LARGE TU Delft}\\[1.5cm] % Name of your university/college
\textsc{\Large EE2821 EPO-3 Tussenrapport}\\[0.5cm] % Major heading such as course name
\textsc{\large Specificaties subsystemen en vooropzet hoofdrapport}\\[0.5cm] % Minor heading such as course title

%----------------------------------------------------------------------------------------
%	TITLE SECTION
%----------------------------------------------------------------------------------------

\HRule \\[0.4cm]
{ \huge \bfseries Realisatie EPO-3 project}\\[0.2cm] % Title of your document
\HRule \\[1.5cm]
 
%----------------------------------------------------------------------------------------
%	AUTHOR SECTION
%----------------------------------------------------------------------------------------

\begin{minipage}{0.4\textwidth}
\begin{flushleft} \large
\emph{Leden:}\\
\begin{tabular}{cc}
Efraïm\textsc{ Eland} & 4170563 \\
Erwin \textsc{de Haan} & 4222814\\
Robin \textsc{Hes}& 4236815\\
Tu \textsc{Hoang} &4203496 \\
Kees \textsc{Hogenhout} & 4220927\\
Alex \textsc{Janssen} &	4231333\\
Jorden \textsc{ Kerkhof} & 4232461\\
Peter \textsc{Stijnman} & 4215788 \\
Xenia \textsc{Wesdijk} & 4144074\\
\end{tabular}

\end{flushleft}
\end{minipage}
~
\begin{minipage}{0.4\textwidth}
\begin{flushright} \large
\emph{Supervisor:} \\
Dr. Marjam \textsc{Bartek} % Supervisor's Name
\end{flushright}
\end{minipage}\\[3cm]

%\Large \emph{Author:}\\
%John \textsc{Smith}\\[1.5cm] % Your name

%----------------------------------------------------------------------------------------
%	DATE SECTION
%----------------------------------------------------------------------------------------

{\large \today}\\ [1.5cm]

%----------------------------------------------------------------------------------------
%	LOGO SECTION
%----------------------------------------------------------------------------------------

%\includegraphics{Logo}\\ 
 
%----------------------------------------------------------------------------------------
% 

\textbf{Abstract} \\
In het EPO-3 project moet er een chip ontworpen worden. Wij hebben ervoor gekozen om als project een GPU te gaan ontwerpen. Ons eindproduct bestaat in dit project uit een transistorlayout voor de chip en een test plan waarin staat hoe de chip getest kan worden.
In dit verslag geven we een beschrijving van hoe we het proces hebben aangepakt. Allereerst zullen we een concept geven van onze GPU waarin we de randvoorwaarden en andere systeemeisen vastleggen. Om het ontwerpproces wat te vergemakkelijken (en om iedereen wat te doen te geven) hebben we het systeem opgedeeld in meerdere deelsystemen. We zullen na het concept dus per deelsysteem een specifieker concept geven. We zullen per deelsysteem de bijbehorende specificaties, functionaliteit en implementatie bespreken. Tevens zullen we per deelsysteem aangeven wat we hebben bereikt en wat er niet goed ging. Verder zullen we de samenhang van de deelsystemen bespreken en hoe dit dus het ontwerp van de chip vormt. Daarbij leggen we ook uit hoe de integratie van de deelsystemen loopt en welke complicaties er zijn opgetreden. Daarna komt er een discussie over de bereikte resultaten. Als laatste komt conclusie. Hierin zullen we de belangrijkste conclusies bespreken en de daarbij behorende resultaten. !!!!! Hier dan ook nog die belangrijkste conclusie neerzetten!!!!!!

\vfill % Fill the rest of the page with whitespace

\end{titlepage}
\end{document}