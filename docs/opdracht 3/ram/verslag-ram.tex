\documentclass{scrartcl}  % scrartcl of scrreprt
% Include all project wide packages here.
\usepackage{fullpage}
\usepackage{polyglossia}
\setmainlanguage{dutch}
\usepackage{csquotes}
\usepackage{graphicx}
\usepackage{epstopdf}
\usepackage{pdfpages}
\usepackage{caption}
\usepackage[list=true]{subcaption}
\usepackage{float}
\usepackage{standalone}
\usepackage{import}
\usepackage{tocloft}
\usepackage{wrapfig}
\usepackage{authblk}
\usepackage{array}
\usepackage{booktabs}
\usepackage[toc,page,title,titletoc]{appendix}
\usepackage{xunicode}
\usepackage{fontspec}
\usepackage{pgfplots}
\usepackage{SIunits}
\pgfplotsset{compat=newest}
\pgfplotsset{plot coordinates/math parser=false}
\newlength\figureheight 
\newlength\figurewidth
\usepackage{unicode-math}
\usepackage[
    backend=bibtexu,
	texencoding=utf8,
bibencoding=utf8,
    style=ieee,
    sortlocale=nl_NL,
    language=auto
]{biblatex}
\usepackage{listings}
\newcommand{\includecode}[3][c]{\lstinputlisting[caption=#2, escapechar=, style=#1]{#3}}
\newcommand{\superscript}[1]{\ensuremath{^{\textrm{#1}}}}
\newcommand{\subscript}[1]{\ensuremath{_{\textrm{#1}}}}


\newcommand{\chapternumber}{\thechapter}
\renewcommand{\appendixname}{Bijlage}
\renewcommand{\appendixtocname}{Bijlagen}
\renewcommand{\appendixpagename}{Bijlagen}

\usepackage[hidelinks]{hyperref} %<--------ALTIJD ALS LAATSTE
  
\renewcommand{\familydefault}{\sfdefault}

\setmainfont[Ligatures=TeX]{Myriad Pro}
\setmathfont{Asana Math}
\setmonofont{Lucida Console}

\usepackage{titlesec, blindtext, color}
\definecolor{gray75}{gray}{0.75}
\newcommand{\hsp}{\hspace{20pt}}
\titleformat{\chapter}[hang]{\Huge\bfseries}{\chapternumber\hsp\textcolor{gray75}{|}\hsp}{0pt}{\Huge\bfseries}
\renewcommand{\familydefault}{\sfdefault}
\renewcommand{\arraystretch}{1.2}
\setlength\parindent{0pt}

%For code listings
\definecolor{black}{rgb}{0,0,0}
\definecolor{browntags}{rgb}{0.65,0.1,0.1}
\definecolor{bluestrings}{rgb}{0,0,1}
\definecolor{graycomments}{rgb}{0.4,0.4,0.4}
\definecolor{redkeywords}{rgb}{1,0,0}
\definecolor{bluekeywords}{rgb}{0.13,0.13,0.8}
\definecolor{greencomments}{rgb}{0,0.5,0}
\definecolor{redstrings}{rgb}{0.9,0,0}
\definecolor{purpleidentifiers}{rgb}{0.01,0,0.01}


\lstdefinestyle{csharp}{
language=[Sharp]C,
showspaces=false,
showtabs=false,
breaklines=true,
showstringspaces=false,
breakatwhitespace=true,
escapeinside={(*@}{@*)},
columns=fullflexible,
commentstyle=\color{greencomments},
keywordstyle=\color{bluekeywords}\bfseries,
stringstyle=\color{redstrings},
identifierstyle=\color{purpleidentifiers},
basicstyle=\ttfamily\small}

\lstdefinestyle{c}{
language=C,
showspaces=false,
showtabs=false,
breaklines=true,
showstringspaces=false,
breakatwhitespace=true,
escapeinside={(*@}{@*)},
columns=fullflexible,
commentstyle=\color{greencomments},
keywordstyle=\color{bluekeywords}\bfseries,
stringstyle=\color{redstrings},
identifierstyle=\color{purpleidentifiers},
}

\lstdefinestyle{matlab}{
language=Matlab,
showspaces=false,
showtabs=false,
breaklines=true,
showstringspaces=false,
breakatwhitespace=true,
escapeinside={(*@}{@*)},
columns=fullflexible,
commentstyle=\color{greencomments},
keywordstyle=\color{bluekeywords}\bfseries,
stringstyle=\color{redstrings},
identifierstyle=\color{purpleidentifiers}
}

\lstdefinestyle{vhdl}{
language=VHDL,
showspaces=false,
showtabs=false,
breaklines=true,
showstringspaces=false,
breakatwhitespace=true,
escapeinside={(*@}{@*)},
columns=fullflexible,
commentstyle=\color{greencomments},
keywordstyle=\color{bluekeywords}\bfseries,
stringstyle=\color{redstrings},
identifierstyle=\color{purpleidentifiers}
}

\lstdefinestyle{xaml}{
language=XML,
showspaces=false,
showtabs=false,
breaklines=true,
showstringspaces=false,
breakatwhitespace=true,
escapeinside={(*@}{@*)},
columns=fullflexible,
commentstyle=\color{greencomments},
keywordstyle=\color{redkeywords},
stringstyle=\color{bluestrings},
tagstyle=\color{browntags},
morestring=[b]",
  morecomment=[s]{<?}{?>},
  morekeywords={xmlns,version,typex:AsyncRecords,x:Arguments,x:Boolean,x:Byte,x:Char,x:Class,x:ClassAttributes,x:ClassModifier,x:Code,x:ConnectionId,x:Decimal,x:Double,x:FactoryMethod,x:FieldModifier,x:Int16,x:Int32,x:Int64,x:Key,x:Members,x:Name,x:Object,x:Property,x:Shared,x:Single,x:String,x:Subclass,x:SynchronousMode,x:TimeSpan,x:TypeArguments,x:Uid,x:Uri,x:XData,Grid.Column,Grid.ColumnSpan,Click,ClipToBounds,Content,DropDownOpened,FontSize,Foreground,Header,Height,HorizontalAlignment,HorizontalContentAlignment,IsCancel,IsDefault,IsEnabled,IsSelected,Margin,MinHeight,MinWidth,Padding,SnapsToDevicePixels,Target,TextWrapping,Title,VerticalAlignment,VerticalContentAlignment,Width,WindowStartupLocation,Binding,Mode,OneWay,xmlns:x}
}

%defaults
\lstset{
basicstyle=\ttfamily\small,
extendedchars=false,
numbers=left,
numberstyle=\ttfamily\tiny,
stepnumber=1,
tabsize=4,
numbersep=5pt
}
\addbibresource{../../library/bibliography.bib}

\author{Erwin {de Haan} (4222814) \and {Tu Hoang} (xxxxxxxx)}
\title{EPO3-1 - Opdracht 3: Moduleontwerp RAM}
\subtitle{In opdracht van Kees Hogenhout (xxxxxxxx) en Jorden Kerhof (xxxxxxxx)}
\date{28 September 2013}

\begin{document}
\maketitle
\pagenumbering{roman}
\section*{Samenvatting}
In dit rapport staat een korte beschrijving van het ontwerpproces van onze RAM module voor op een Sea-of-Gates chip.
Wij hebben een complete behavioural RAM gemaakt en daarna een RAM opgebouwd uit zorgvuldig gemaakte subcellen.
\newpage
\setlength{\cftbeforetoctitleskip}{-3em}
\tableofcontents

\section{Inleiding}
%TODO inleiding
Wij moesten aan de hand van specificatie gopgesteld door andere leden van onze project groep een RAM module implementeren.
De hebben wij gedaan door gebruik te maken van onder andere VHDL en GoWithTheFlow. \nocite{epo3-manual}

\newpage
\pagenumbering{arabic}

\subimport{}{specificaties-ram.tex}

\subsection{Veranderingen}
Wij hebben de reset (full ram clear) niet geimplementeerd omdat dit in een RAM niet nodig is en heel veel ruimte en logica bespaart.
Ook hebben we een 2-bits input voor het commando (read/write) vervangen door een enkele inputlijn die aangeeft of er geschreven moet worden.
Als laatste hebben wij een ready (output) lijn eruit geoptimaliseerd.

\section{Ontwerp}
De woorden hebben we als losse entiteiten gezien die worden dus in één keer geschreven of gelezen.
Het totaal aantal bit is dus een veelvoud van de woord grootte.
Als het adres veranderd switch de multiplexor voor de uitgang meteen naar het correcte woord, de data staat dus na een korte vertraging meteen op de uitgang ook als er nieuwe data geschreven wordt.
De nieuwe data wordt naar het goede word geleid net al het write signaal, dus alleen het woord dat aangewezen wordt door het adres krijgt de opdracht om te schrijven.
Het blackbox model dat volgt uit de specificatie staat in Figuur \ref{fig:blackbox-ram}.
\begin{figure}[H]
\centering
		\includegraphics[width=\textwidth]{resource/blackbox-ram}
		\caption{Het blackbox model van ons RAM met de ingangen en uitgangen.}
		\label{fig:blackbox-ram}
\end{figure}
Ons ontwerp is opgedeeld in verschillende blokken omdat onze eerste compleet behavioural beschrijving niet lekker te routen was met een hoge efficientie. Zie figuur \ref{fig:subsystems-ram}.
De kleinste blokjes zijn ramword’s dit is gewoon een “stapel” flip-flop’s.
Een rambank combineert al deze ramword’s.
De rammux zijn gewoon een tweetal grote selectie multiplexers en een kleine die het write signaal goed verdeeld.
\begin{figure}[H]
\centering
		\includegraphics[width=\textwidth]{resource/subsystems-ram}
		\caption{De subsystemen en de sommige tussenverbindingen.}
		\label{fig:subsystems-ram}
\end{figure}
De VHDL implementatie is te zien in Bijlage \ref{app:ram-vhdl}. De implementatie van onze eerste versie is te zien in Bijlage \ref{app:b-ram-vhdl}.

%TODO sim results modelsim

\section{Implementatie}
De eerste implementatie hadden we vrij snel klaar. De tweede en meer stucturele hebben we veel langer over gedaan, zelfs zo lang dat we uit eindelijk in tijdnood kwamen. De sysnthse heeft ons veel problemen gegeven vooral de synthese van de grote multiplexers. Maar uit eindelijk hebben we alle gesynthetiseerd gekregen. De layout hebben we waar mogelijk met de hand gedaan. Het routen van de multiplexers naar de rambank gaf het meeste problemen dit is ons uiteindelijk ook niet gelukt.
%TODO switch level sim and layout

\section{Conclusies}
We voldoen volledig eaan de specificaties.
De data staat na <10ns stabiel op de uitgang.
Het duurt maximaal 1 klokslag om data op te slaan.
Iemand kan om snel vier woorden vol te schrijven, de write input pin hoog maken en dan in de zelfde klokslag het eerste adres en het eerste woord aanbieden.
Daarna kan iedere volgende klokslag een nieuw adres en woord aangeboden worden.
We zouden meer aandacht kunnen besteden aan de multiplexer structuul omdat onze rambank op zich zelf heeft een excellente efficientie van 100 procent maar de multiplexer maakte dat het alleen te routen was met een efficientie van rond de 30 procent.
Hier kan dus nog aan gewerkt worden.

\newpage
\pagenumbering{Roman}

\printbibliography
\renewcommand{\chapternumber}{\appendixname\;\thechapter}
\begin{appendices}
\section{VHDL implementatie RAM}
\label{app:b-ram-vhdl}
\subsection{ram-pkg.vhd}
\label{appsec:b-ram-pkg.vhd}
\includecode[vhdl]{ram-pkg.vhd}{../../../hardware/opdracht3/ram/VHDL/ram_pkg.vhd}
\subsection{ram.vhd}
\label{appsec:b-ram.vhd}
\includecode[vhdl]{ram.vhd}{../../../hardware/opdracht3/ram/VHDL/ram.vhd}
\subsection{ram-behaviour.vhd}
\label{appsec:b-ram-behaviour.vhd}
\includecode[vhdl]{ram-behaviour.vhd}{../../../hardware/opdracht3/ram/VHDL/ram-behaviour.vhd}

\section{VHDL implementatie RAM2}
\label{app:ram-vhdl}
\subsection{ram-pkg.vhd}
\label{appsec:ram-pkg.vhd}
\includecode[vhdl]{ram-pkg.vhd}{../../../hardware/opdracht3/ram2/VHDL/ram_pkg.vhd}

\subsection{ram.vhd}
\label{appsec:ram.vhd}
\includecode[vhdl]{ram.vhd}{../../../hardware/opdracht3/ram2/VHDL/ram.vhd}

\subsection{ram-behaviour.vhd}
\label{appsec:ram-behaviour.vhd}
\includecode[vhdl]{ram-behaviour.vhd}{../../../hardware/opdracht3/ram2/VHDL/ram-behaviour.vhd}

\subsection{rammux.vhd}
\label{appsec:rammux.vhd}
\includecode[vhdl]{rammux.vhd}{../../../hardware/opdracht3/ram2/VHDL/rammux.vhd}

\subsection{rammux-behaviour.vhd}
\label{appsec:rammux-behaviour.vhd}
\includecode[vhdl]{rammux-behaviour.vhd}{../../../hardware/opdracht3/ram2/VHDL/rammux-behaviour.vhd}

\subsection{rambank.vhd}
\label{appsec:rambank.vhd}
\includecode[vhdl]{rambank.vhd}{../../../hardware/opdracht3/ram2/VHDL/rambank.vhd}

\subsection{rambank-behaviour.vhd}
\label{appsec:rambank-behaviour.vhd}
\includecode[vhdl]{rambank-behaviour.vhd}{../../../hardware/opdracht3/ram2/VHDL/rambank-behaviour.vhd}

\subsection{ramword.vhd}
\label{appsec:ramword.vhd}
\includecode[vhdl]{ramword.vhd}{../../../hardware/opdracht3/ram2/VHDL/ramword.vhd}

\subsection{ramword-behaviour.vhd}
\label{appsec:ramword-behaviour.vhd}
\includecode[vhdl]{ramword-behaviour.vhd}{../../../hardware/opdracht3/ram2/VHDL/ramword-behaviour.vhd}
\end{appendices}
\end{document}