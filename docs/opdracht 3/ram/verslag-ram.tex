\documentclass{scrartcl}  % scrartcl of scrreprt
% Include all project wide packages here.
\usepackage{fullpage}
\usepackage{polyglossia}
\setmainlanguage{dutch}
\usepackage{csquotes}
\usepackage{graphicx}
\usepackage{epstopdf}
\usepackage{pdfpages}
\usepackage{caption}
\usepackage[list=true]{subcaption}
\usepackage{float}
\usepackage{standalone}
\usepackage{import}
\usepackage{tocloft}
\usepackage{wrapfig}
\usepackage{authblk}
\usepackage{array}
\usepackage{booktabs}
\usepackage[toc,page,title,titletoc]{appendix}
\usepackage{xunicode}
\usepackage{fontspec}
\usepackage{pgfplots}
\usepackage{SIunits}
\pgfplotsset{compat=newest}
\pgfplotsset{plot coordinates/math parser=false}
\newlength\figureheight 
\newlength\figurewidth
\usepackage{unicode-math}
\usepackage[
    backend=bibtexu,
	texencoding=utf8,
bibencoding=utf8,
    style=ieee,
    sortlocale=nl_NL,
    language=auto
]{biblatex}
\usepackage{listings}
\newcommand{\includecode}[3][c]{\lstinputlisting[caption=#2, escapechar=, style=#1]{#3}}
\newcommand{\superscript}[1]{\ensuremath{^{\textrm{#1}}}}
\newcommand{\subscript}[1]{\ensuremath{_{\textrm{#1}}}}


\newcommand{\chapternumber}{\thechapter}
\renewcommand{\appendixname}{Bijlage}
\renewcommand{\appendixtocname}{Bijlagen}
\renewcommand{\appendixpagename}{Bijlagen}

\usepackage[hidelinks]{hyperref} %<--------ALTIJD ALS LAATSTE
  
\renewcommand{\familydefault}{\sfdefault}

\setmainfont[Ligatures=TeX]{Myriad Pro}
\setmathfont{Asana Math}
\setmonofont{Lucida Console}

\usepackage{titlesec, blindtext, color}
\definecolor{gray75}{gray}{0.75}
\newcommand{\hsp}{\hspace{20pt}}
\titleformat{\chapter}[hang]{\Huge\bfseries}{\chapternumber\hsp\textcolor{gray75}{|}\hsp}{0pt}{\Huge\bfseries}
\renewcommand{\familydefault}{\sfdefault}
\renewcommand{\arraystretch}{1.2}
\setlength\parindent{0pt}

%For code listings
\definecolor{black}{rgb}{0,0,0}
\definecolor{browntags}{rgb}{0.65,0.1,0.1}
\definecolor{bluestrings}{rgb}{0,0,1}
\definecolor{graycomments}{rgb}{0.4,0.4,0.4}
\definecolor{redkeywords}{rgb}{1,0,0}
\definecolor{bluekeywords}{rgb}{0.13,0.13,0.8}
\definecolor{greencomments}{rgb}{0,0.5,0}
\definecolor{redstrings}{rgb}{0.9,0,0}
\definecolor{purpleidentifiers}{rgb}{0.01,0,0.01}


\lstdefinestyle{csharp}{
language=[Sharp]C,
showspaces=false,
showtabs=false,
breaklines=true,
showstringspaces=false,
breakatwhitespace=true,
escapeinside={(*@}{@*)},
columns=fullflexible,
commentstyle=\color{greencomments},
keywordstyle=\color{bluekeywords}\bfseries,
stringstyle=\color{redstrings},
identifierstyle=\color{purpleidentifiers},
basicstyle=\ttfamily\small}

\lstdefinestyle{c}{
language=C,
showspaces=false,
showtabs=false,
breaklines=true,
showstringspaces=false,
breakatwhitespace=true,
escapeinside={(*@}{@*)},
columns=fullflexible,
commentstyle=\color{greencomments},
keywordstyle=\color{bluekeywords}\bfseries,
stringstyle=\color{redstrings},
identifierstyle=\color{purpleidentifiers},
}

\lstdefinestyle{matlab}{
language=Matlab,
showspaces=false,
showtabs=false,
breaklines=true,
showstringspaces=false,
breakatwhitespace=true,
escapeinside={(*@}{@*)},
columns=fullflexible,
commentstyle=\color{greencomments},
keywordstyle=\color{bluekeywords}\bfseries,
stringstyle=\color{redstrings},
identifierstyle=\color{purpleidentifiers}
}

\lstdefinestyle{vhdl}{
language=VHDL,
showspaces=false,
showtabs=false,
breaklines=true,
showstringspaces=false,
breakatwhitespace=true,
escapeinside={(*@}{@*)},
columns=fullflexible,
commentstyle=\color{greencomments},
keywordstyle=\color{bluekeywords}\bfseries,
stringstyle=\color{redstrings},
identifierstyle=\color{purpleidentifiers}
}

\lstdefinestyle{xaml}{
language=XML,
showspaces=false,
showtabs=false,
breaklines=true,
showstringspaces=false,
breakatwhitespace=true,
escapeinside={(*@}{@*)},
columns=fullflexible,
commentstyle=\color{greencomments},
keywordstyle=\color{redkeywords},
stringstyle=\color{bluestrings},
tagstyle=\color{browntags},
morestring=[b]",
  morecomment=[s]{<?}{?>},
  morekeywords={xmlns,version,typex:AsyncRecords,x:Arguments,x:Boolean,x:Byte,x:Char,x:Class,x:ClassAttributes,x:ClassModifier,x:Code,x:ConnectionId,x:Decimal,x:Double,x:FactoryMethod,x:FieldModifier,x:Int16,x:Int32,x:Int64,x:Key,x:Members,x:Name,x:Object,x:Property,x:Shared,x:Single,x:String,x:Subclass,x:SynchronousMode,x:TimeSpan,x:TypeArguments,x:Uid,x:Uri,x:XData,Grid.Column,Grid.ColumnSpan,Click,ClipToBounds,Content,DropDownOpened,FontSize,Foreground,Header,Height,HorizontalAlignment,HorizontalContentAlignment,IsCancel,IsDefault,IsEnabled,IsSelected,Margin,MinHeight,MinWidth,Padding,SnapsToDevicePixels,Target,TextWrapping,Title,VerticalAlignment,VerticalContentAlignment,Width,WindowStartupLocation,Binding,Mode,OneWay,xmlns:x}
}

%defaults
\lstset{
basicstyle=\ttfamily\small,
extendedchars=false,
numbers=left,
numberstyle=\ttfamily\tiny,
stepnumber=1,
tabsize=4,
numbersep=5pt
}
\addbibresource{../../library/bibliography.bib}

\author{Erwin {de Haan} (4222814) \and {Tu Hoang} (4203496)}
\title{EPO3-1 - Opdracht 3: Moduleontwerp RAM}
\subtitle{In opdracht van Kees Hogenhout (xxxxxxxx) en Jorden Kerhof (3458923)}
\date{28 September 2013}

\begin{document}
\maketitle
\pagenumbering{roman}
\section*{Samenvatting}
In dit rapport staat een korte beschrijving van het ontwerpproces van onze RAM module voor op een Sea-of-Gates chip.
Wij hebben een complete behavioural RAM gemaakt en daarna een RAM opgebouwd uit zorgvuldig gemaakte subcellen.
\newpage
\setlength{\cftbeforetoctitleskip}{-3em}
\tableofcontents

\section{Inleiding}
%TODO inleiding
Wij moesten aan de hand van specificaties opgesteld door andere leden van onze project groep een RAM module implementeren.
Dit hebben wij gedaan door gebruik te maken van onder andere VHDL en GoWithTheFlow.

\newpage
\pagenumbering{arabic}

\subimport{}{specificaties-ram.tex}

\subsection{Veranderingen}
Wij hebben de reset (full ram clear) niet geimplementeerd omdat dit in een RAM niet nodig is en heel veel ruimte en logica bespaart.
Ook hebben we een 2-bits input voor het commando (read/write) vervangen door een enkele inputlijn die aangeeft of er geschreven moet worden.
Als laatste hebben wij een ready (output) lijn eruit geoptimaliseerd.

\section{Ontwerp}
De woorden hebben we als losse entiteiten gezien die dus in één keer worden geschreven of gelezen.
Het totaal aantal bit is dus een veelvoud van de woord grootte.
Als het adres verandert, switcht de multiplexer voor de uitgang meteen naar het correcte woord, de data staat dus na een korte vertraging meteen op de uitgang ook als er nieuwe data geschreven wordt.
De nieuwe data wordt naar het goede woord geleid net al het write signaal, dus alleen het woord dat aangewezen wordt door het adres krijgt de opdracht om te schrijven.
Het blackbox model dat volgt uit de specificatie staat in Figuur \ref{fig:blackbox-ram}.
\begin{figure}[H]
\centering
		\includegraphics[width=\textwidth]{resource/blackbox-ram}
		\caption{Het blackbox model van ons RAM met de ingangen en uitgangen.}
		\label{fig:blackbox-ram}
\end{figure}
Ons ontwerp is opgedeeld in verschillende blokken, omdat onze eerste complete behavioural beschrijving niet goed te routen was met een hoge efficientie. Zie figuur \ref{fig:subsystems-ram}.
De kleinste blokjes zijn ramword’s dit is gewoon een “stapel” flip-flop’s.
Een rambank combineert al deze ramword’s.
De rammux zijn gewoon een tweetal grote selectie multiplexers en een kleine die het write signaal goed verdeeld.
\begin{figure}[H]
\centering
		\includegraphics[width=\textwidth]{resource/subsystems-ram}
		\caption{De subsystemen en de sommige tussenverbindingen.}
		\label{fig:subsystems-ram}
\end{figure}
De VHDL implementatie is te zien in Bijlage \ref{app:ram-vhdl}. De implementatie van onze eerste versie is te zien in Bijlage \ref{app:b-ram-vhdl}.

\begin{figure}[H]
\centering
		\includegraphics[width=\textwidth]{resource/modelsimsim}
		\caption{De simulatie resultaten uit modelsim.}
		\label{fig:modelsim-ram}
\end{figure}

\section{Implementatie}
De eerste implementatie hadden we vrij snel klaar. De tweede en meer stucturele hebben we veel langer over gedaan, zelfs zo lang dat we uit eindelijk in tijdnood kwamen. De sysnthese heeft ons veel problemen gegeven vooral de synthese van de grote multiplexers. Maar uit eindelijk hebben we alles gesynthetiseerd gekregen. De layout hebben we waar mogelijk met de hand gedaan. Het routen van de multiplexers naar de rambank gaf het meeste problemen, maar dit is ons uiteindelijk wel gelukt.
\begin{figure}[H]
\centering
		\includegraphics[height=\textwidth, angle=90]{resource/ramword}
		\caption{Layout van component ramword. (90 graden gedraaid)}
		\label{fig:layout-ramword}
\end{figure}
\begin{figure}[H]
\centering
		\includegraphics[width=\textwidth]{resource/rambank}
		\caption{Layout van component rambank.}
		\label{fig:layout-rambank}
\end{figure}
\begin{figure}[H]
\centering
		\includegraphics[width=\textwidth]{resource/rammux}
		\caption{Layout van component rammux.}
		\label{fig:layout-rammux}
\end{figure}
\begin{figure}[H]
\centering
		\includegraphics[width=\textwidth]{resource/ram}
		\caption{Layout van het RAM.}
		\label{fig:layout-ram}
\end{figure}
\begin{figure}[H]
\centering
		\includegraphics[width=0.5\textwidth]{resource/compare}
		\caption{De resultaten van de simulatie vergelijking.}
		\label{fig:compare-ram}
\end{figure}

\section{Conclusies}
We voldoen volledig aan de specificaties.
Wij hebben onze RAM gemaakt met 8 woorden van 8 bits.
De data staat na <10ns stabiel op de uitgang.
Het duurt maximaal 1 klokslag om data op te slaan.
Iemand kan om snel vier woorden vol te schrijven, de write input pin hoog maken en dan in de zelfde klokslag het eerste adres en het eerste woord aanbieden.
Daarna kan iedere volgende klokslag een nieuw adres en woord aangeboden worden.
We zouden meer aandacht kunnen besteden aan de multiplexer structuur en plaating van de terminals, omdat onze rambank heeft op zichzelf een excellente efficientie van 100 procent.
De multiplexer maakte dat het alleen te routen was met een efficientie van rond de 26.17 procent.
Er worden 4376 van de 16720 transistoren gebruikt van het gebruikte oppervlakte in ons ontwerp.
Wij hebben 100000 transistoren tot onze beschikking. \cite[93]{epo3-manual}
Als de vorm dat toelaat zouden we in theorie dus $\frac{100000}{16720} \approx 5.9 \rightarrow 5$ RAMs kunnen gebruiken als we bijna geen andere logica zouden gebruiken.
Als we dit als één RAM zouden willen implementeren wordt het pas echt een probleem.
Dan word het maken van de correcte multiplexers echt een probleem.
Men zou moeten kijken of men alle inputs van de rambank aan de bovenkant en de outputs aan de onderkant kan plaatsen.
Als met dan de input en output multiplexer splitst en deze in twee relatief platte cellen te maken om deze dichtbij de inputs en output van de rambank te plaatsten.
Het is dan belangrijk dat de pint dezelfde volgorde hebben.
Hier kan dus nog aan gewerkt worden.

\section{Discussie}
Het ontworpen circuit en daarbij met name de efficiëntie kan veel verbeterd worden. 
Vooral door de beperkende software van place \& route is de efficiëntie veel lager dan het had kunnen zijn. 
Bij het maken van de rammux layout kan het programma niet op een uitvoerbare manier de onderdelen plaatsen, waardoor we de rammux zoveel chip ruimte moesten geven dat het maar ongeveer 60 procent efficiënt was.
De plaatsing van de in- en uitgangen van de subcellen wordt bepaald door het programma en het heeft de in- en uitgangen op willekeurige plaatsen op de rand van de cel geplaatst, dit heeft ervoor gezorgd dat we de rambank en rammux cellen midden op de chip moesten plaatsen om ruimte te geven voor de bedrading van de desbetreffende in- en uitgangen, zoals je op fig:layout-ram kan zien.
Om het toch zo efficiënt mogelijk te maken moet er gedacht worden aan de plaatsing van de multiplexer om zo de multiplexer in- en uitgangen bij het desbetreffende woord te plaatsen, zo zal de software hier een betere plaatsing mee kunnen maken en zal het uiteindelijke resultaat een efficiëntere ram zijn met een beter mogelijkheid tot uitbreiding. 
\pagenumbering{Roman}

\printbibliography
\renewcommand{\chapternumber}{\appendixname\;\thechapter}
\begin{appendices}
\section{VHDL implementatie RAM}
\label{app:b-ram-vhdl}
\subsection{ram-pkg.vhd}
\label{appsec:b-ram-pkg.vhd}
\includecode[vhdl]{ram-pkg.vhd}{../../../hardware/opdracht3/ram/VHDL/ram_pkg.vhd}
\subsection{ram.vhd}
\label{appsec:b-ram.vhd}
\includecode[vhdl]{ram.vhd}{../../../hardware/opdracht3/ram/VHDL/ram.vhd}
\subsection{ram-behaviour.vhd}
\label{appsec:b-ram-behaviour.vhd}
\includecode[vhdl]{ram-behaviour.vhd}{../../../hardware/opdracht3/ram/VHDL/ram-behaviour.vhd}

\section{VHDL implementatie RAM2}
\label{app:ram-vhdl}
\subsection{ram-pkg.vhd}
\label{appsec:ram-pkg.vhd}
\includecode[vhdl]{ram-pkg.vhd}{../../../hardware/opdracht3/ram2/VHDL/ram_pkg.vhd}

\subsection{ram.vhd}
\label{appsec:ram.vhd}
\includecode[vhdl]{ram.vhd}{../../../hardware/opdracht3/ram2/VHDL/ram.vhd}

\subsection{ram-behaviour.vhd}
\label{appsec:ram-behaviour.vhd}
\includecode[vhdl]{ram-behaviour.vhd}{../../../hardware/opdracht3/ram2/VHDL/ram-behaviour.vhd}

\subsection{rammux.vhd}
\label{appsec:rammux.vhd}
\includecode[vhdl]{rammux.vhd}{../../../hardware/opdracht3/ram2/VHDL/rammux.vhd}

\subsection{rammux-behaviour.vhd}
\label{appsec:rammux-behaviour.vhd}
\includecode[vhdl]{rammux-behaviour.vhd}{../../../hardware/opdracht3/ram2/VHDL/rammux-behaviour.vhd}

\subsection{rambank.vhd}
\label{appsec:rambank.vhd}
\includecode[vhdl]{rambank.vhd}{../../../hardware/opdracht3/ram2/VHDL/rambank.vhd}

\subsection{rambank-behaviour.vhd}
\label{appsec:rambank-behaviour.vhd}
\includecode[vhdl]{rambank-behaviour.vhd}{../../../hardware/opdracht3/ram2/VHDL/rambank-behaviour.vhd}

\subsection{ramword.vhd}
\label{appsec:ramword.vhd}
\includecode[vhdl]{ramword.vhd}{../../../hardware/opdracht3/ram2/VHDL/ramword.vhd}

\subsection{ramword-behaviour.vhd}
\label{appsec:ramword-behaviour.vhd}
\includecode[vhdl]{ramword-behaviour.vhd}{../../../hardware/opdracht3/ram2/VHDL/ramword-behaviour.vhd}
\end{appendices}
\end{document}
