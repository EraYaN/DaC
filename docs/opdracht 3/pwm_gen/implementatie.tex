%=========================================================================
\documentclass[11pt,twoside,a4paper]{article}
\usepackage[dutch]{babel}
\usepackage{a4wide,times}


\begin{document}



\section{Implementatie}

Behavioural\\
Als allereerste is de VHDL-code geschreven, het idee en de opzet voor de code waren snel geschreven en daar zaten nog een paar overbodige gedeeltes in die we er later uitgehaald hebben.
Ook na het simuleren, het resultaat is hieronder te zien in figuur..., bleek alles volges verwachting en konden we verder met de synthese.\\

Synthese\\
Na het simuleren van de gesynthetiseerde VHDL voor de eerste keer bleek dat er drops waren in het uitgangssignaal van ongeveer 10 ns, op een klokperiode van 100 ns. 
Dit bleek alleen te gebeuren als de bitsize van de counter te groot werd. Hierdoor wordt namelijk de logische schakeling te groot en krijg je last van de gate delays.
Toen wij eenmaal een flipflop hadden toegevoegd aan het einde van de schakeling, waardoor het de uitgang gebufferd wordt , was dit probleem verholpen. 
De simulatie van de VHDL-code en de gesynthetiseerde VHDL kwam nu overeen en klopte ook met de verwachtingen zoals hieronder is te zien in figuur....\\

Layoutresultaat.\\
Hierna hebben we de layout voor de schakeling gemaakt. Bij lage bitsizes voor de counter kon het programma seadali het nog af met de automatic tools maddona en trout, voor row placement en wiring respectievelijk.
Toen de bitsize van de counter overigens te groot werd hebben zelf de row placement moeten doen omdat anders de wiring niet automatisch goed ging. Uiteindelijk hadden we een efficiency van 51.00% bij onze layout.
Dit getal kan natuurlijk een stuk hoger nog, maar aangezien we er niet voor gekozen hebben om elk blok met de hand er in te zetten maar het programma seadali het automatisch te laten doen. Hier hebben we voor gekozen omdat 
het ten eerste veel tijd kost om te doen en ten tweede omdat we nog niet weten welke bitsize we voor de counter nodig gaan hebben en we dus anders voor bijvoorbeeld 4 verschillende bitsizes dit process mogen gaan doen.
De uiteindelijke layout is hier onder te zien in figuur ...\\

Switch-level simulatieresultaten van de uit de layout geëxtraheerde schakeling.\\
Als laatste hebben we de VHDL-code geëxtraheerd uit de switch-level schakeling. Dit is ter controle of de synthetisering goed is gegaan of dat er fouten zitten in de layout.
Deze fouten kunnen ontstaan door de eerder besproken gate delays, gelukkig hadden na onze goede synthese hier geen last van. Hier waren we dus geen fouten tegen gekomen.
Ook hier zijn de resultaten van de simulatie hieronder te vinden in figuur....\\






\end{document}
