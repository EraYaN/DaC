%=========================================================================
\documentclass[11pt,twoside,a4paper]{article}
% Include all project wide packages here.
\usepackage{fullpage}
\usepackage{polyglossia}
\setmainlanguage{dutch}
\usepackage{csquotes}
\usepackage{graphicx}
\usepackage{epstopdf}
\usepackage{pdfpages}
\usepackage{caption}
\usepackage[list=true]{subcaption}
\usepackage{float}
\usepackage{standalone}
\usepackage{import}
\usepackage{tocloft}
\usepackage{wrapfig}
\usepackage{authblk}
\usepackage{array}
\usepackage{booktabs}
\usepackage[toc,page,title,titletoc]{appendix}
\usepackage{xunicode}
\usepackage{fontspec}
\usepackage{pgfplots}
\usepackage{SIunits}
\pgfplotsset{compat=newest}
\pgfplotsset{plot coordinates/math parser=false}
\newlength\figureheight 
\newlength\figurewidth
\usepackage{unicode-math}
\usepackage[
    backend=bibtexu,
	texencoding=utf8,
bibencoding=utf8,
    style=ieee,
    sortlocale=nl_NL,
    language=auto
]{biblatex}
\usepackage{listings}
\newcommand{\includecode}[3][c]{\lstinputlisting[caption=#2, escapechar=, style=#1]{#3}}
\newcommand{\superscript}[1]{\ensuremath{^{\textrm{#1}}}}
\newcommand{\subscript}[1]{\ensuremath{_{\textrm{#1}}}}


\newcommand{\chapternumber}{\thechapter}
\renewcommand{\appendixname}{Bijlage}
\renewcommand{\appendixtocname}{Bijlagen}
\renewcommand{\appendixpagename}{Bijlagen}

\usepackage[hidelinks]{hyperref} %<--------ALTIJD ALS LAATSTE

\renewcommand{\familydefault}{\sfdefault}

\setmainfont[Ligatures=TeX]{Myriad Pro}
\setmathfont{Asana Math}
\setmonofont{Lucida Console}

\usepackage{titlesec, blindtext, color}
\definecolor{gray75}{gray}{0.75}
\newcommand{\hsp}{\hspace{20pt}}
\titleformat{\chapter}[hang]{\Huge\bfseries}{\chapternumber\hsp\textcolor{gray75}{|}\hsp}{0pt}{\Huge\bfseries}
\renewcommand{\familydefault}{\sfdefault}
\renewcommand{\arraystretch}{1.2}
\setlength\parindent{0pt}

%For code listings
\definecolor{black}{rgb}{0,0,0}
\definecolor{browntags}{rgb}{0.65,0.1,0.1}
\definecolor{bluestrings}{rgb}{0,0,1}
\definecolor{graycomments}{rgb}{0.4,0.4,0.4}
\definecolor{redkeywords}{rgb}{1,0,0}
\definecolor{bluekeywords}{rgb}{0.13,0.13,0.8}
\definecolor{greencomments}{rgb}{0,0.5,0}
\definecolor{redstrings}{rgb}{0.9,0,0}
\definecolor{purpleidentifiers}{rgb}{0.01,0,0.01}


\lstdefinestyle{csharp}{
language=[Sharp]C,
showspaces=false,
showtabs=false,
breaklines=true,
showstringspaces=false,
breakatwhitespace=true,
escapeinside={(*@}{@*)},
columns=fullflexible,
commentstyle=\color{greencomments},
keywordstyle=\color{bluekeywords}\bfseries,
stringstyle=\color{redstrings},
identifierstyle=\color{purpleidentifiers},
basicstyle=\ttfamily\small}

\lstdefinestyle{c}{
language=C,
showspaces=false,
showtabs=false,
breaklines=true,
showstringspaces=false,
breakatwhitespace=true,
escapeinside={(*@}{@*)},
columns=fullflexible,
commentstyle=\color{greencomments},
keywordstyle=\color{bluekeywords}\bfseries,
stringstyle=\color{redstrings},
identifierstyle=\color{purpleidentifiers},
}

\lstdefinestyle{matlab}{
language=Matlab,
showspaces=false,
showtabs=false,
breaklines=true,
showstringspaces=false,
breakatwhitespace=true,
escapeinside={(*@}{@*)},
columns=fullflexible,
commentstyle=\color{greencomments},
keywordstyle=\color{bluekeywords}\bfseries,
stringstyle=\color{redstrings},
identifierstyle=\color{purpleidentifiers}
}

\lstdefinestyle{vhdl}{
language=VHDL,
showspaces=false,
showtabs=false,
breaklines=true,
showstringspaces=false,
breakatwhitespace=true,
escapeinside={(*@}{@*)},
columns=fullflexible,
commentstyle=\color{greencomments},
keywordstyle=\color{bluekeywords}\bfseries,
stringstyle=\color{redstrings},
identifierstyle=\color{purpleidentifiers}
}

\lstdefinestyle{xaml}{
language=XML,
showspaces=false,
showtabs=false,
breaklines=true,
showstringspaces=false,
breakatwhitespace=true,
escapeinside={(*@}{@*)},
columns=fullflexible,
commentstyle=\color{greencomments},
keywordstyle=\color{redkeywords},
stringstyle=\color{bluestrings},
tagstyle=\color{browntags},
morestring=[b]",
  morecomment=[s]{<?}{?>},
  morekeywords={xmlns,version,typex:AsyncRecords,x:Arguments,x:Boolean,x:Byte,x:Char,x:Class,x:ClassAttributes,x:ClassModifier,x:Code,x:ConnectionId,x:Decimal,x:Double,x:FactoryMethod,x:FieldModifier,x:Int16,x:Int32,x:Int64,x:Key,x:Members,x:Name,x:Object,x:Property,x:Shared,x:Single,x:String,x:Subclass,x:SynchronousMode,x:TimeSpan,x:TypeArguments,x:Uid,x:Uri,x:XData,Grid.Column,Grid.ColumnSpan,Click,ClipToBounds,Content,DropDownOpened,FontSize,Foreground,Header,Height,HorizontalAlignment,HorizontalContentAlignment,IsCancel,IsDefault,IsEnabled,IsSelected,Margin,MinHeight,MinWidth,Padding,SnapsToDevicePixels,Target,TextWrapping,Title,VerticalAlignment,VerticalContentAlignment,Width,WindowStartupLocation,Binding,Mode,OneWay,xmlns:x}
}

%defaults
\lstset{
basicstyle=\ttfamily\small,
extendedchars=false,
numbers=left,
numberstyle=\ttfamily\tiny,
stepnumber=1,
tabsize=4,
numbersep=5pt
}

%\usepackage{a4wide,times}
\usepackage{fullpage}
\usepackage{verbatim}
\title{Moduleopdracht ALU}
\author{
Gemaakt door\\
Jorden, 3458923\\
Kees Hogenhout, \\
\\
In opdracht van \\
Efraïm, \\
Alex, \\
}
\date{\today}

\begin{document}
\maketitle
\thispagestyle{empty}
\vspace{30 mm}
\begin{center}
\Large \bf 
Samenvatting
\end{center}
In dit rapport wordt beschreven hoe een groepje van twee studenten een module ontwerpt, in dit geval een ALU. Daarbij werden de specificaties bepaald door een ander groepje van twee in de projectgroep. Er moest minimaal 1 parameter bepaald worden door het andere groepje. De specificatie in dit rapport is het aantal bits input en output.
 \\
Vervolgens  wordt de aan het groepje opgedragen schakeling ontworpen en getest op VHDL behavior niveau, met verwisselende parameterwaarden. Daarna wordt de schakeling geïmplementeerd tot op layout niveau, inclusief switch-level simulatie. 
Daarna worden er in het rapport simulatieresultaten getoond, om aan te tonen dat het circuit werkt. 
\clearpage

\tableofcontents
\clearpage

\section{Inleiding}
Bij deze opdracht was het de bedoeling om in kleine groepjes van de projectgroep afzonderlijk een module te ontwerpen. De specificaties werden gegeven door een ander groepje. Aan ons was de taak gegeven een Alu te ontwerpen. Verdere specificaties van deze Alu worden hieronder vermeld. Van deze  alu hebben we een ontwerp gemaakt in vhdl en gesimuleerd. Vervolgens hebben we hiervan een circuit gemodelleerd en chip lay-out gecreëerd. 

\section{Specificaties}
\scriptsize
\verbatiminput{ALU_beschrijving.txt}
\normalsize

\section{Ontwerp}

%\begin{itemize}
%\item
%Blokrepresentatie met in en uitgangen
%\item
%Eventuele opdeling in deelschakelingen
%\item
%Een of meerdere toestandsdiagrammen 
%\item
%VHDL behaviour beschrijving van de schakeling.
%\item
%Simulatieresultaten van de VHDL behaviour beschrijving.
%\end{itemize}

Ons GPU is ingedeeld volgens het volgende  blokschema.

\subsection{SPI Decoder}

\subsection{Instruction Decoder}

\subsection{Draw Logic}

\subsection{RAM Controller}

\subsection{State Registers}

\subsection{VGA Controller}



\section{Implementatie}
%=========================================================================
\documentclass[11pt,twoside,a4paper]{article}
\usepackage[dutch]{babel}
\usepackage{a4wide,times}


\begin{document}



\section{Implementatie}

Behavioural\\
Als allereerste is de VHDL-code geschreven, het idee en de opzet voor de code waren snel geschreven en daar zaten nog een paar overbodige gedeeltes in die we er later uitgehaald hebben.
Ook na het simuleren, het resultaat is hieronder te zien in figuur..., bleek alles volges verwachting en konden we verder met de synthese.\\

Synthese\\
Na het simuleren van de gesynthetiseerde VHDL voor de eerste keer bleek dat er drops waren in het uitgangssignaal van ongeveer 10 ns, op een klokperiode van 100 ns. 
Dit bleek alleen te gebeuren als de bitsize van de counter te groot werd. Hierdoor wordt namelijk de logische schakeling te groot en krijg je last van de gate delays.
Toen wij eenmaal een flipflop hadden toegevoegd aan het einde van de schakeling, waardoor het de uitgang gebufferd wordt , was dit probleem verholpen. 
De simulatie van de VHDL-code en de gesynthetiseerde VHDL kwam nu overeen en klopte ook met de verwachtingen zoals hieronder is te zien in figuur....\\

Layoutresultaat.\\
Hierna hebben we de layout voor de schakeling gemaakt. Bij lage bitsizes voor de counter kon het programma seadali het nog af met de automatic tools maddona en trout, voor row placement en wiring respectievelijk.
Toen de bitsize van de counter overigens te groot werd hebben zelf de row placement moeten doen omdat anders de wiring niet automatisch goed ging. Uiteindelijk hadden we een efficiency van 51.00% bij onze layout.
Dit getal kan natuurlijk een stuk hoger nog, maar aangezien we er niet voor gekozen hebben om elk blok met de hand er in te zetten maar het programma seadali het automatisch te laten doen. Hier hebben we voor gekozen omdat 
het ten eerste veel tijd kost om te doen en ten tweede omdat we nog niet weten welke bitsize we voor de counter nodig gaan hebben en we dus anders voor bijvoorbeeld 4 verschillende bitsizes dit process mogen gaan doen.
De uiteindelijke layout is hier onder te zien in figuur ...\\

Switch-level simulatieresultaten van de uit de layout geëxtraheerde schakeling.\\
Als laatste hebben we de VHDL-code geëxtraheerd uit de switch-level schakeling. Dit is ter controle of de synthetisering goed is gegaan of dat er fouten zitten in de layout.
Deze fouten kunnen ontstaan door de eerder besproken gate delays, gelukkig hadden na onze goede synthese hier geen last van. Hier waren we dus geen fouten tegen gekomen.
Ook hier zijn de resultaten van de simulatie hieronder te vinden in figuur....\\






\end{document}


\section{Conclusies}
%De conclusie is zeer belangrijk. Deze is meestal relatief kort, en vat heel beknopt de resultaten en
%belangrijkste reflectie daarop nog eens samen. Wanneer de conclusie te lang zou worden, is het beter
%om een aparte discussie sectie (zie hierboven) toe te voegen. In de conclusie kun je, indien van
%toepassing, aanbevelingen doen voor verder werk/onderzoek. (Indien dat te lang zou worden, kun je
%daar ook weer een aparte sectie van maken.) In de conclusie maak je ook eventuele slotopmerkingen.
%Hou er rekening mee dat er lezers zijn die alleen het abstract en de conclusie lezen, en maak deze
%stukken ook voor hen nuttig, bijvoorbeeld om te beslissen of ze het gehele rapport gaan lezen


De probleemstelling bevatte kernachtig het volgende: Bepaal de dikte van het gateoxide van de NMOS transistor. De dikte is bepaald via de capaciteit van het polysilicium. Deze dikte is voor de PMOS 25.0 nanometer en voor de NMOS 25.3 nanometer. Deze diktes zijn bepaald met het computerprogramma SPICE en niet met echte transistors. Een aanbeveling is om ook echte transistors door te meten, als dat mogelijk is. \\ \newline
Hoe kijken we als groepje tegen de simulatie? Eerst hebben we erg lang allerlei circuits geprobeerd, waarna we erachter kwamen dat we de goede grafiek eigenlijk al hadden. Hij was alleen nog niet te beredeneren en dat is nu wel het geval. Deze grafieken vormden de basis voor het uitrekenen van de capaciteiten door het invullen van de functie in SPICE. Verder was het handig om matlab te gebruiken bij het maken van grafieken en bij het afleiden van data uit die grafieken. (basic linear fitting) Daarna was het nog wel even lastig om een gestructureerd verslag in elkaar te zetten, nu was immers geen template beschikbaar. Al met al is de opdracht geslaagd. 


Voldoet het ontwerp aan de specificatie ?
Wat zou er eventueel verbeterd kunnen worden ?

\begin{thebibliography}{7} %je kunt maximaal 9 verwijzingen hebben
%Auteur, A. (jaar van uitgave). textit{Titel van het boek. Plaats: Uitgeverij.
%Apa normen
%
\bibitem 
HHendriks E.A., Bakker A., Frehe A., Groeneveld P., Nouta R., Verhoeven C., Graaf S. de , Liedorp J., Genderen A.J. van,  (2013) \textit{Projecthandleiding Ontwerp een chip}, Uitgegeven in Delft

\bibitem 
BBrown Stephen, Vranesic Zvonko (2009), \textit{Fundamentals of Digital Logic with VHDL Design}, Published by McGraw-Hill Education



\end{thebibliography}






\newpage

\section{Appendix}
\appendix
\section{entity alu}
\label{entity}
\lstinputlisting{../../../hardware/opdracht3/alu3/VHDL/alu.vhd}


\section{behaviour of alu}
\label{behaviour}
\lstinputlisting{../../../hardware/opdracht3/alu3/VHDL/alu-behaviour.vhd}

\section{behaviour of tb}
\label{tb}
\lstinputlisting{../../../hardware/opdracht3/alu3/VHDL/tb-behaviour.vhd}



\end{document}
