%=========================================================================
\documentclass[11pt,twoside,a4paper]{article}
\usepackage[dutch]{babel}
%\usepackage{a4wide,times}
\title{Moduleopdracht ALU}
\author{
Gemaakt door\\
Jorden, 3458923\\
Kees Hogenhout, \\
\\
In opdracht van \\
Efraïm, \\
Alex, \\
}
\date{\today}

\begin{document}
\maketitle1
\thispagestyle{empty}
\vspace{30 mm}

\begin{center}
\Large \bf 
Samenvatting
\end{center}
In dit rapport wordt beschreven hoe een groepje van twee studenten een module ontwerpt, in dit geval een ALU. Daarbij werden de specificaties bepaald door een ander groepje van twee in de projectgroep. Er moest minimaal 1 parameter bepaald worden door het andere groepje. De specificatie in dit rapport is het aantal bits input en output. \newline \\
Vervolgens  wordt de aan het groepje opgedragen schakeling ontworpen en getest op VHDL behavior niveau, met verwisselende parameterwaarden. Daarna wordt de schakeling geïmplementeerd tot op layout niveau, inclusief switch-level simulatie. 
Daarna worden er in het rapport simulatieresultaten getoond, om aan te tonen dat het circuit werkt. 
\clearpage

\tableofcontents
\clearpage

\section{Inleiding}
Vertel i

\section{Specificatie}
\input{..\}

\section{Ontwerp}
Beschrijf hoe de beschrijving van de opdrachtgevers verder geïnterpreteerd is en hoe het ontwerp gedaan is, tot aan de VHDL beschrijving.
\begin{itemize}
\item
Blokrepresentatie met in en uitgangen
\item
Eventuele opdeling in deelschakelingen
\item
Een of meerdere toestandsdiagrammen 
\item
VHDL behaviour beschrijving van de schakeling.
\item
Simulatieresultaten van de VHDL behaviour beschrijving.
\end{itemize}

\section{Implementatie}
Vertel over de implementatie van de schakeling:
\begin{itemize}
\item
Hoe ging de synthese ?
\item
Layoutresultaat.
\item
Switch-level/Spice simulatieresultaten van de uit de layout ge-extraheerde schakeling.
\end{itemize}

\section{Conclusies}
Voldoet het ontwerp aan de specificatie ?
Wat zou er eventueel verbeterd kunnen worden ?

\begin{thebibliography}{9}
\bibitem{labelboek1}
Auteur1, 
Boek1, 
Uitgeverij, 
jaar.
\bibitem{labelTitel}
Auteur1, 
Titel, 
Tijdschrift, 
Vol. 1, 
Nr. 1, 
jaar, 
pp. 12-15.
\bibitem{labelWeb}
Webpagina titel, 
http link, 
geraadpleegd op 19 sep. 2013.
\end{thebibliography}
\end{document}
