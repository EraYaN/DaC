\documentclass{scrartcl} % scrartcl of scrreprt
% Include all project wide packages here.
\usepackage{fullpage}
\usepackage{polyglossia}
\setmainlanguage{dutch}
\usepackage{csquotes}
\usepackage{graphicx}
\usepackage{epstopdf}
\usepackage{pdfpages}
\usepackage{caption}
\usepackage[list=true]{subcaption}
\usepackage{float}
%\usepackage{mathtools}
\usepackage{standalone}
\usepackage{import}
\usepackage{tocloft}
\usepackage{wrapfig}
\usepackage{authblk}
\usepackage{array}
\usepackage{booktabs}
\usepackage[toc,page,title,titletoc]{appendix}
\usepackage{xunicode}
\usepackage{amsmath}
\usepackage{fontspec}
\usepackage{unicode-math}
\usepackage[
    backend=bibtexu,
	texencoding=utf8,
bibencoding=utf8,
    style=ieee,
    sortlocale=nl_NL,
    language=auto
]{biblatex}
\usepackage{listings}
\newcommand{\includecode}[3][c]{\lstinputlisting[caption=#2, escapechar=, style=#1]{#3}}
\newcommand{\superscript}[1]{\ensuremath{^{\textrm{#1}}}}
\newcommand{\subscript}[1]{\ensuremath{_{\textrm{#1}}}}


\newcommand{\chapternumber}{\thechapter}
\renewcommand{\appendixname}{Bijlage}
\renewcommand{\appendixtocname}{Bijlagen}
\renewcommand{\appendixpagename}{Bijlagen}

\usepackage[hidelinks]{hyperref} %<--------ALTIJD ALS LAATSTE

\renewcommand{\familydefault}{\sfdefault}

\setmainfont[Ligatures=TeX]{Myriad Pro}
\setmathfont{Asana Math}
\setmonofont{Lucida Console}

\usepackage{titlesec, blindtext, color}
\definecolor{gray75}{gray}{0.75}
\newcommand{\hsp}{\hspace{20pt}}
\titleformat{\chapter}[hang]{\Huge\bfseries}{\chapternumber\hsp\textcolor{gray75}{|}\hsp}{0pt}{\Huge\bfseries}
\renewcommand{\familydefault}{\sfdefault}
\renewcommand{\arraystretch}{1.2}
\setlength\parindent{0pt}

%For code listings
\definecolor{black}{rgb}{0,0,0}
\definecolor{browntags}{rgb}{0.65,0.1,0.1}
\definecolor{bluestrings}{rgb}{0,0,1}
\definecolor{graycomments}{rgb}{0.4,0.4,0.4}
\definecolor{redkeywords}{rgb}{1,0,0}
\definecolor{bluekeywords}{rgb}{0.13,0.13,0.8}
\definecolor{greencomments}{rgb}{0,0.5,0}
\definecolor{redstrings}{rgb}{0.9,0,0}
\definecolor{purpleidentifiers}{rgb}{0.01,0,0.01}


\lstdefinestyle{csharp}{
language=[Sharp]C,
showspaces=false,
showtabs=false,
breaklines=true,
showstringspaces=false,
breakatwhitespace=true,
escapeinside={(*@}{@*)},
columns=fullflexible,
commentstyle=\color{greencomments},
keywordstyle=\color{bluekeywords}\bfseries,
stringstyle=\color{redstrings},
identifierstyle=\color{purpleidentifiers},
basicstyle=\ttfamily\small}

\lstdefinestyle{c}{
language=C,
showspaces=false,
showtabs=false,
breaklines=true,
showstringspaces=false,
breakatwhitespace=true,
escapeinside={(*@}{@*)},
columns=fullflexible,
commentstyle=\color{greencomments},
keywordstyle=\color{bluekeywords}\bfseries,
stringstyle=\color{bluestrings},
identifierstyle=\color{purpleidentifiers}
}

\lstdefinestyle{vhdl}{
language=VHDL,
showspaces=false,
showtabs=false,
breaklines=true,
showstringspaces=false,
breakatwhitespace=true,
escapeinside={(*@}{@*)},
columns=fullflexible,
commentstyle=\color{greencomments},
keywordstyle=\color{bluekeywords}\bfseries,
stringstyle=\color{redstrings},
identifierstyle=\color{purpleidentifiers}
}

\lstdefinestyle{xaml}{
language=XML,
showspaces=false,
showtabs=false,
breaklines=true,
showstringspaces=false,
breakatwhitespace=true,
escapeinside={(*@}{@*)},
columns=fullflexible,
commentstyle=\color{greencomments},
keywordstyle=\color{redkeywords},
stringstyle=\color{bluestrings},
tagstyle=\color{browntags},
morestring=[b]",
  morecomment=[s]{<?}{?>},
  morekeywords={xmlns,version,typex:AsyncRecords,x:Arguments,x:Boolean,x:Byte,x:Char,x:Class,x:ClassAttributes,x:ClassModifier,x:Code,x:ConnectionId,x:Decimal,x:Double,x:FactoryMethod,x:FieldModifier,x:Int16,x:Int32,x:Int64,x:Key,x:Members,x:Name,x:Object,x:Property,x:Shared,x:Single,x:String,x:Subclass,x:SynchronousMode,x:TimeSpan,x:TypeArguments,x:Uid,x:Uri,x:XData,Grid.Column,Grid.ColumnSpan,Click,ClipToBounds,Content,DropDownOpened,FontSize,Foreground,Header,Height,HorizontalAlignment,HorizontalContentAlignment,IsCancel,IsDefault,IsEnabled,IsSelected,Margin,MinHeight,MinWidth,Padding,SnapsToDevicePixels,Target,TextWrapping,Title,VerticalAlignment,VerticalContentAlignment,Width,WindowStartupLocation,Binding,Mode,OneWay,xmlns:x}
}

%defaults
\lstset{
basicstyle=\ttfamily\small,
extendedchars=false,
numbers=left,
numberstyle=\ttfamily\tiny,
stepnumber=1,
tabsize=4,
numbersep=5pt
}
\addbibresource{../../library/bibliography.bib}

\author{Peter Stijnman}
\title{EPO3: Eindrapport - Procesverloop}

\begin{document}
\chapter{Procesverloop}
\label{ch:procesverloop}

\subsection{ Planning}

Voor dit project moesten we zelf een strakke planning maken zodat we het op tijd af zouden krijgen. Tijdens het plannen gingen we ervan uit dat we minimaal 3 lessen aan het einde van het project nodig zouden hebben om eventuele problemen nog op te lossen
en voor het geval we zouden uitlopen op de planning. We hadden gepland dat we alle modules in VHDL af zouden hebben binnen 3 lessen en dat we dan konden gaan testen en synthetyseren. Dit omdat er tijdens dat process heel veel fouten en bugs naar voren zouden komen.
Dit was ook het geval en voor het correct synthetyseren hadden we ook weer anderhalve week voor uitgetrokken. De over gebleven tijd zou gebruikt worden voor het testen van meerdere modules aan elkaar en het complete systeem.\\
Het idee was dat het meeste verslag schrijven thuis zou gebeuren omdat hierbij niet veel dingen fout kunnen gaan en met het programmeren wel en we dan hulp kunnen vragen aan andere. Mocht tijdens de tijd van de project les de module al goed werken en er verder niks 
anders te doen was kon er natuurlijk wel aan het verslag worden geschreven.\\ De extra onderdelen die niet bij de chip hoorde, de arduïno en het externe RAM, werd ook zoveel mogelijk thuis voorbereid om dan tijdens de les meer te kunnen testen. Verder als er dingen niet 
tijdens de les af waren was het de bedoeling om het thuis af te maken voor de volgende keer. Aan deze planning heeft iedereen zich gehouden en dat heeft geholpen met de realisatie van de chip.

\subsection{Implementeren}

Het implementeren van de aparte modules in VHDL ging vrij soepel. Dit omdat iedereen al vaker met VHDL-code had gewerkt, dit bracht dus ook niet veel problemen. Alleen een aantal keer dat er extra signalen moesten worden toegevoegd voor communicatie tussen 
verschillende modules van de chip. Verder werd er alleen tijdens het implementeren nog wat kleine veranderingen gemaakt, een voorbeeld hiervan is dat de stateregisters eruit werden gehaald en er omdat deze toch niet nodig bleken te zijn. 

\subsection{Synthetyseren}

Na het implementeren van de modules werden ze allemaal gesynthetyseerd, dit zorgde af en toe nog wel voor problemen met het aantal gebruikte transisitoren dat veels te hoog was voor wat het eigenlijk mocht opnemen. Plus op het switchlevel niveau zaten er af en toe 
wat bugs die eruit gehaald moesten worden. Ook kwamen we op dit niveau van testen er achter dat sommige onderdelen te groot waren om te realiseren. In eerste instantie wouden we een algoritme implementeren dat cirkels kon tekenen. Na het synthetyseren bleek dat dit 
meer dan 10.000 transistor paren zou in gaan nemen en daar hadden we geen ruimte voor. Daarom hebben we de functie voor cirkels geschrapt van het ontwerp.


\subsection{Extra onderdelen}

Het realiseren van de 2 extra onderdelen, de arduïno en het externe RAM, hebben we buiten de project uren gedaan. Dit omdat het niet bij de projectomschrijving hoort en dus extra werk is. Deze onderdelen zijn alleen tijdens de projecturen getest in combinatie met de chip. Voor het verbeteren van de communicatie tussen de externe onderdelen en de chip zal er tijdens de kerstvakantie hoogst waarschijnlijk nog gewerkt worden aan de arduïno en het externe RAM. Vooral de arduïno kan veel tijd in beslag nemen, aangezien hier alle opdrachten vandaan komen en dat betekent ook dat de spelletjes die we willen testen hierop geprogrammeerd moeten worden. Voor zover bekend is zal in ieder geval het spel Snake geïmplementeerd worden en waarschijnlijk ook Pong.

\subsection{Verslag schrijven}

Het verslag schrijven werd opgedeeld naar de hoeveelheid werk iemand in de les en thuis al moest doen. De mensen die de arduïno hebben geprogrammeerd en met het FPGA thuis hebben getest hoefde minder verslag te schrijven dan mensen die in de les hun modules konden
implementeren en synthetyseren. Ook mensen die ziek of afwezig waren kregen een groter stuk verslag om te schrijven aangezien ze veel van de software niet thuis hebben en omdat ze daar minders snel hulp kunnen vragen aan andere mensen.\\
Dit leek ons een eerlijke verdeling qua werk. Omdat iedereen zo thuis ook nog aan het project kon en moest werken en het niet alleen maar door 3 of 4 mensen gedaan moest worden. Bovendien werkt het ook beter als iedereen aan een verslag schrijft en ermee bezig is, je hebt zo meer mensen voor spellingscontrole en überhaupt mensen die het verslag nakijken en misschien onderdelen missen die er nog bij moeten.




\end{document}
