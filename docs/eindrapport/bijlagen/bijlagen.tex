\documentclass{scrreprt}
% Include all project wide packages here.
\usepackage{fullpage}
\usepackage{polyglossia}
\setmainlanguage{dutch}
\usepackage{csquotes}
\usepackage{graphicx}
\usepackage{epstopdf}
\usepackage{pdfpages}
\usepackage{caption}
\usepackage[list=true]{subcaption}
\usepackage{float}
\usepackage{standalone}
\usepackage{import}
\usepackage{tocloft}
\usepackage{wrapfig}
\usepackage{authblk}
\usepackage{array}
\usepackage{booktabs}
\usepackage[toc,page,title,titletoc]{appendix}
\usepackage{xunicode}
\usepackage{fontspec}
\usepackage{pgfplots}
\usepackage{SIunits}
\pgfplotsset{compat=newest}
\pgfplotsset{plot coordinates/math parser=false}
\newlength\figureheight 
\newlength\figurewidth
\usepackage{unicode-math}
\usepackage[
    backend=bibtexu,
	texencoding=utf8,
bibencoding=utf8,
    style=ieee,
    sortlocale=nl_NL,
    language=auto
]{biblatex}
\usepackage{listings}
\newcommand{\includecode}[3][c]{\lstinputlisting[caption=#2, escapechar=, style=#1]{#3}}
\newcommand{\superscript}[1]{\ensuremath{^{\textrm{#1}}}}
\newcommand{\subscript}[1]{\ensuremath{_{\textrm{#1}}}}


\newcommand{\chapternumber}{\thechapter}
\renewcommand{\appendixname}{Bijlage}
\renewcommand{\appendixtocname}{Bijlagen}
\renewcommand{\appendixpagename}{Bijlagen}

\usepackage[hidelinks]{hyperref} %<--------ALTIJD ALS LAATSTE

\renewcommand{\familydefault}{\sfdefault}

\setmainfont[Ligatures=TeX]{Myriad Pro}
\setmathfont{Asana Math}
\setmonofont{Lucida Console}

\usepackage{titlesec, blindtext, color}
\definecolor{gray75}{gray}{0.75}
\newcommand{\hsp}{\hspace{20pt}}
\titleformat{\chapter}[hang]{\Huge\bfseries}{\chapternumber\hsp\textcolor{gray75}{|}\hsp}{0pt}{\Huge\bfseries}
\renewcommand{\familydefault}{\sfdefault}
\renewcommand{\arraystretch}{1.2}
\setlength\parindent{0pt}

%For code listings
\definecolor{black}{rgb}{0,0,0}
\definecolor{browntags}{rgb}{0.65,0.1,0.1}
\definecolor{bluestrings}{rgb}{0,0,1}
\definecolor{graycomments}{rgb}{0.4,0.4,0.4}
\definecolor{redkeywords}{rgb}{1,0,0}
\definecolor{bluekeywords}{rgb}{0.13,0.13,0.8}
\definecolor{greencomments}{rgb}{0,0.5,0}
\definecolor{redstrings}{rgb}{0.9,0,0}
\definecolor{purpleidentifiers}{rgb}{0.01,0,0.01}


\lstdefinestyle{csharp}{
language=[Sharp]C,
showspaces=false,
showtabs=false,
breaklines=true,
showstringspaces=false,
breakatwhitespace=true,
escapeinside={(*@}{@*)},
columns=fullflexible,
commentstyle=\color{greencomments},
keywordstyle=\color{bluekeywords}\bfseries,
stringstyle=\color{redstrings},
identifierstyle=\color{purpleidentifiers},
basicstyle=\ttfamily\small}

\lstdefinestyle{c}{
language=C,
showspaces=false,
showtabs=false,
breaklines=true,
showstringspaces=false,
breakatwhitespace=true,
escapeinside={(*@}{@*)},
columns=fullflexible,
commentstyle=\color{greencomments},
keywordstyle=\color{bluekeywords}\bfseries,
stringstyle=\color{redstrings},
identifierstyle=\color{purpleidentifiers},
}

\lstdefinestyle{matlab}{
language=Matlab,
showspaces=false,
showtabs=false,
breaklines=true,
showstringspaces=false,
breakatwhitespace=true,
escapeinside={(*@}{@*)},
columns=fullflexible,
commentstyle=\color{greencomments},
keywordstyle=\color{bluekeywords}\bfseries,
stringstyle=\color{redstrings},
identifierstyle=\color{purpleidentifiers}
}

\lstdefinestyle{vhdl}{
language=VHDL,
showspaces=false,
showtabs=false,
breaklines=true,
showstringspaces=false,
breakatwhitespace=true,
escapeinside={(*@}{@*)},
columns=fullflexible,
commentstyle=\color{greencomments},
keywordstyle=\color{bluekeywords}\bfseries,
stringstyle=\color{redstrings},
identifierstyle=\color{purpleidentifiers}
}

\lstdefinestyle{xaml}{
language=XML,
showspaces=false,
showtabs=false,
breaklines=true,
showstringspaces=false,
breakatwhitespace=true,
escapeinside={(*@}{@*)},
columns=fullflexible,
commentstyle=\color{greencomments},
keywordstyle=\color{redkeywords},
stringstyle=\color{bluestrings},
tagstyle=\color{browntags},
morestring=[b]",
  morecomment=[s]{<?}{?>},
  morekeywords={xmlns,version,typex:AsyncRecords,x:Arguments,x:Boolean,x:Byte,x:Char,x:Class,x:ClassAttributes,x:ClassModifier,x:Code,x:ConnectionId,x:Decimal,x:Double,x:FactoryMethod,x:FieldModifier,x:Int16,x:Int32,x:Int64,x:Key,x:Members,x:Name,x:Object,x:Property,x:Shared,x:Single,x:String,x:Subclass,x:SynchronousMode,x:TimeSpan,x:TypeArguments,x:Uid,x:Uri,x:XData,Grid.Column,Grid.ColumnSpan,Click,ClipToBounds,Content,DropDownOpened,FontSize,Foreground,Header,Height,HorizontalAlignment,HorizontalContentAlignment,IsCancel,IsDefault,IsEnabled,IsSelected,Margin,MinHeight,MinWidth,Padding,SnapsToDevicePixels,Target,TextWrapping,Title,VerticalAlignment,VerticalContentAlignment,Width,WindowStartupLocation,Binding,Mode,OneWay,xmlns:x}
}

%defaults
\lstset{
basicstyle=\ttfamily\small,
extendedchars=false,
numbers=left,
numberstyle=\ttfamily\tiny,
stepnumber=1,
tabsize=4,
numbersep=5pt
}
\addbibresource{../../library/bibliography.bib}

\author{}
\title{EPO3: Eindrapport - Bijlagen}

\begin{document}
\begin{appendices}
\renewcommand{\chapternumber}{\appendixname\space\thechapter} 
\chapter{Parameters}
\label{app:parameters}
\begin{table}[H]
	\centering
	\caption{De gebruikte parameters en generics}
	\label{tab:spec-params}
	\begin{tabular}{l l l p{0.5\textwidth}}
		\hline\hline
	 	 & Naam & Waarde & Omschrijving\\
	 	\hline
		%GLOBAL
		pkg & InstrSize & 3 & De grootte van een instructie\\ 
		pkg & MaxNumPackets & 7 & Maximaal aantal pakketjes in één instructie\\ 
		pkg & SizeSpriteID & 10 & Het aantal bits voor het SpriteID \\
		pkg & SizeSpriteCounter & 6 & Het aantal bits voor de sprite counter\\
		pkg & TimoutCount & $2^{7}-1$ & Het aantal ticks per timout\\
		pkg & SizeTimeoutCounter & 7 & Het aantal bits voor de timout counter\\
		pkg & NumDrawModules & 4 & Aantal Draw modules\\ 
		pkg & SizeX & 8 & Het aantal bits voor de X coordinaat\\
		pkg & SizeY & 7 & Het aantal bits voor de Y coordinaat\\
		pkg & SizeRAMData & 6 & Aantal bits per woord in het RAM\\
		pkg & SizeRAMAddr & 16 & Aantal adresbits voor het RAM\\
		pkg & SizeColor & 6 & Aantal bits kleur\\
		pkg & SizeSPIData & 8 & Aantal bits van de SPI pakketjes\\
		pkg & ResolutionX & 160 & Horizontale resolutie \\
		pkg & ResolutionY & 120 & Verticale resolutie \\
		%VGA
		vga & h\_pulse & 24 & Aantal klokpulsen voor de horizontale syncpuls\\ 
		vga & h\_bp & 12 & Aantal klokpulsen voor de horizontale “backporch” \\ 
		vga & h\_pixels & 160 & Aantal klokpulsen voor de horizontale pixels \\ 
		vga & h\_fp & 4 & Aantal klokpulsen voor de horizontale “fontporch” \\ 
		vga & h\_pol & ‘0’ & De polariteit van de horizontale syncpuls\\ 
		vga & v\_pulse & 2 & Aantal rijen voor de verticale syncpuls\\ 
		vga & v\_bp & 33 & Aantal rijen voor de verticale “backporch” \\ 
		vga & v\_pixels & 480 & Aantal rijen voor de verticale pixels\\ 
		vga & v\_fp & 10 & Aantal rijen voor de verticale “fontporch” \\ 
		vga & v\_pol & ‘0’ & De polariteit van de verticale syncpuls\\ 
		%SPI
		spi & c & 3 & Het aantal bits van de counter in de SPI module\\
		%SPRITE
		sprite & SizeDCounter & 3 & Aantal bits van de datacounter\\
		sprite & DCounterMax & 6 & De maximale waarde van de datacounter\\
		sprite & DCounterStart & 6 & De startwaarde van de DCounter is de laadslag \\
	  	\hline
	\end{tabular}
\end{table}

\chapter{Instructies}
\label{app:instructies}
\begin{table}[H]
	\centering
	\caption{De gebruikte instructies en hun data in ons protocol}
	\label{tab:spec-instr}
	\begin{tabular}{l l p{0.75\textwidth}}
		\hline\hline
	 	\# & Naam & Pakketten (in <octets>)\\
	 	\hline
		0 & SWITCH & 0x00\\ 
		1 & FILL & 0x01 <color>\\ 
		2 & PIXEL & 0x02 <color> <x> <y>\\
		3 & RECT & 0x03 <color> <x0> <y0> <x1> <y1>\\
		4 & FRECT & 0x04 <color> <x0> <y0> <x1> <y1>\\
		5 & LINE & 0x05 <color> <x0> <y0> <x1> <y1>\\
		6 & SPRITE & 0x06 <color> <x> <y> <width in pixels> <6-bit length in addresses + 2 MSB bits off start address> <8 LSB bits of start address>\\ 
		7 & LSPRITE &  0x07 <6-bit length in addresses + 2 MSB bits off start address> <8 LSB bits of start address> then length times <spritedata>\\
	  	\hline
	\end{tabular}
\end{table}

\chapter{Pseudocode}
\label{app:pseudocode}
\section{Bresenham's line algorithm}
\label{appsec:pseudocode-bresenham-line}
\includecode[c]{Pseudocode Bresenham's line algorithm \cite{line-alg}}{resource/bresenham-line.pseudocode}

\chapter{Broncode}
\label{app:source}
\subimport{../../../}{sourceBijlage}

\end{appendices}
\end{document}
